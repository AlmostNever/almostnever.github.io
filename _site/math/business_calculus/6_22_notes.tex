% --------------------------------------------------------------
% Andrew Tindall
% --------------------------------------------------------------
\documentclass[12pt]{article}
% --------------------------------------------------------------
% Andrew Tindall
% --------------------------------------------------------------
 
 
\usepackage[margin=1in]{geometry} 
\usepackage{amsmath,amsthm,amssymb,enumitem,hyperref,tikz-cd, mdframed, fancyhdr}

\newcommand{\N}{\mathbb{N}}
\newcommand{\Q}{\mathbb{Q}}
\newcommand{\Z}{\mathbb{Z}}
\newcommand{\C}{\mathbb{C}}
\newcommand{\R}{\mathbb{R}}
\newcommand{\A}{\mathbb{A}}
\newcommand{\bs}{\backslash}


\theoremstyle{definition}
\newtheorem{theorem}{Theorem}
\newmdtheoremenv{theobox}{Theorem}
\makeatletter
\makeatother


\newenvironment{problem}[2][Problem]{\begin{trivlist}
\item[\hskip \labelsep {\bfseries #1}\hskip \labelsep {\bfseries #2.}]}{\end{trivlist}}
 

\setlength{\jot}{10pt}
\pagestyle{fancy}
\fancyhf{}
\rhead{Math 120 Recitation\\ Summer 6WK2, 2020}
\lhead{Day 1\\
June 22, 2020} 
\begin{document}
\par These notes are based on the transcript of the first meeting of the math 120 business calc recitation. 
\section{Finding Slope-Intercept Form from $2$ points} 
\par If we have a pair of points $(x_1, y_1)$ and $(x_2, y_2)$, there is a unique line that goes through them. A common problem is to find the equation of that line in slope intercept form, which is $y = mx + b$.
\begin{itemize}
\item $y$ is the \textit{independent variable}, or ``a function of'' $x$.
\item $x$ is the \textit{dependent variable}.
\item $m$ is the \textit{slope} of the line.
\item $b$ is the \textit{y-intercept}.
\end{itemize}
What you'll do first is calculate $m$, the slope of the line, with a $\frac{\Delta y}{\Delta x}$ formula, and then calculate $b$, the $y$-intercept, by plugging values into the $y = mx + b$ equation.
\par The slope equation is
\[ m = \frac{\Delta y}{\Delta x} = \frac{y_2 - y_1}{x_2 - x_1}.\]
Read $\Delta$ as ``change in.'' So, $\Delta y$ is $y_2 - y_1$ and $\Delta x$ is $x_2 - x_1$.
\par I'll pick a specific pair of points, and we'll calculate the equation of the line through them, in the form $y = mx + b$.
\par Say, $(1, 3)$ and $(5, 15)$.
\par So, $x_1$ is $1$ and $x_2$ is $5$; and $y_1$ is $3$ and $y_2$ is $15$. Plugging these values into the formula for $m$, we get
\begin{align*}
m &= \frac{\Delta y}{ \Delta x}\\
&= \frac{15 - 3}{5 - 1}\\
&= \frac{12}{4}\\
&= 3
\end{align*}
So, $m = 3$. This tells us that the equation of our line will be \[y = 3x + b.\] 
\par Next, we need to find $b$, the $y$-intercept of the line. To do this, we plug the $x$ and $y$ values of one of the points into $y = 3x + b$. If we use the values from the first point, we get $x = 1$ and $y = 3$:
\begin{align*}
 y &= 3x + b\\
3 &= 3 \cdot 1 + b\\
3 &= 3 + b\\
0 &= b
\end{align*}
So, $b = 0$. If we substitute $b = 0$ into $y = 3x + b$, we get the equation
\[ y = 3x + 0,\]
or just
\[y = 3x.\]
So, this line will have a pretty steep slope of $3$, and it will go through the $y$-intercept $(0,0)$.
\section{Applying the Slope Intercept Form}
Say a wireless providor has a linear price for your data usage. So, they have a flat starting rate for a month of service, and then they charge the same amount for each gigabyte of data usage.
\par Now, say they charge \$75 for $5$ GB, and \$85 for $7$ GB. We want to find the price, \$$P$, as a function of the amount of data, $d$ GB.
\par The first step here is really to translate the numbers in the problem into points on a graph. So, we need to identify what the independent and dependent variables are.
\par The price you pay is \textit{dependent} on the number of gigabytes of data. We can also say that we're finding $P$ as a function of $d$. 
\par Since $d$ is the independent variable, it goes where $x$ would go in the equation $y = mx + b$, and because $P$ is the dependent variable, it goes where $y$ would go.
\begin{align*}
y &= mx + b\\
&\Downarrow\\
P &= md + b
\end{align*}
Our two points will be $(5, 75)$ and $(7, 85)$. We want to find the equation for the line between these two points, the same way we found the equation for the line through the last two points. Step $1$ is to find the slope $m$ using the $\frac{\Delta y}{\Delta x}$ formula:
\begin{align*}
 m &= \frac{\Delta P}{\Delta d}\\
&= \frac{85 - 75}{7 - 5}\\
&= \frac{10}{2}\\
&= 5.
\end{align*}
So, the slope of the graph is $5$. A slope is a \textit{rate}, and it is always given in units of $y$ divided by units of $x$. Here, the units of $y$ are dollars, and the units of $d$ are gigabytes, so the rate of this plan is $5$  \textit{dollars per gigabyte}. \footnote{If a unit is a ratio, like miles divided by hours, it is common to write it as miles \textit{per} hour.} 
\par Now that we know that our slope $m$ is $5$, we can find the $y$-intercept (or ``$P$-intercept'') $b$. Again, we can plug one of the points into the equation $P = 5d + b$ to find $b$. Let's use the point $(7, 85)$. Here, $P = 85$ and $d = 7$:
\begin{align*}
 P &= 5d + b\\
85 &= 5 \cdot 7 + b\\
85 &= 35 + b\\
50 &= b.
\end{align*}
So, the $y$-intercept for this line is $50$, and the equation of $P$ as a function of $d$ that we were looking for is 
\[P = 5d + 50.\]
\section{Finding Values of a Linear Function}
We can use the equation we just found, to find the price of using any given number of gigabytes of data. So, say, what's the cost of using $10$ gigabytes?
\par To solve this, we simply subsistute $d = 10$ into the equation:
\begin{align*}
P &= 5d + 50\\
P &= 5 \cdot 10 + 50\\
P &= 50 + 50\\
P&= 100
\end{align*}
So, the cost of using $10$ gigabytes in a month is \$$100$.
\par In the opposite direction, we can use the equation to find $d$ from a value of $P$. For example, how much data would you be using in a month, given that your plan costs \$125?
\par Here, we are given the value $P = 125$, and we can substitute it into our equation:
\begin{align*}
P &= 5d + 50\\
125 &= 5d + 50\\
75 &= 5d\\
15 &= d
\end{align*}
So, to pay $125$ dollars a month, we would need to use exactly $15$ gigabytes.
\section{}<++>
\end{document}

