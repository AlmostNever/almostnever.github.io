% --------------------------------------------------------------
% Andrew Tindall
% --------------------------------------------------------------
 
\documentclass[12pt]{article}
 
\usepackage[margin=1in]{geometry} 
\usepackage{amsmath,amsthm,amssymb,enumitem,hyperref}

\newcommand{\N}{\mathbb{N}}
\newcommand{\Q}{\mathbb{Q}}
\newcommand{\Z}{\mathbb{Z}}
\newcommand{\C}{\mathbb{C}}
\newcommand{\R}{\mathbb{R}}
\newcommand{\mc}[1]{\mathcal{#1}}
\newcommand{\e}{\varepsilon}
\newcommand{\bs}{\backslash}
\newcommand{\PGL}{\text{PGL}}
\newcommand{\Sp}{\text{Sp}}
\newcommand{\tr}{\text{tr}}
\newcommand{\Lie}{\text{Lie}}
\newcommand{\rec}[1]{\frac{1}{#1}}
\newcommand{\toinf}{\rightarrow \infty}


\theoremstyle{definition}
\newtheorem{proofpart}{Part}
\newtheorem{theorem}{Theorem}
\makeatletter
\@addtoreset{proofpart}{theorem}
\makeatother


\newenvironment{problem}[2][Problem]{\begin{trivlist}
\item[\hskip \labelsep {\bfseries #1}\hskip \labelsep {\bfseries #2.}]}{\end{trivlist}}
 
\begin{document}
 
%\renewcommand{\qedsymbol}{\filledbox}
 
\title{Homework 1}
\author{Andrew Tindall\\
Topology II}
\maketitle
\begin{section}{Problems}
	\begin{problem}{1}
		Prove that $S^n$ is homeomorphic to the quotient topological space $D^n/\sim$, where $\sim$ denotes the following equivalence relation on $D^n$:
		\[x \sim y \iff x = y \text{ or } x \text{ and } y \in \partial D^n. \]
		Let $S^n$ be the set of unit vectors in $\R^{n+1}$: $S^n = \left\{ x \in \R^{n+1}; \left \lvert { x } \right \lvert  = 1 \right\}$.
		\begin{proof}
			We first construct a map $\varphi: S^n \to D^n / \sim$. Let $x \in S^{n} = \langle x_1, \dots x_{n}, x_{n+1}\rangle$. Let $u(x_1, \dots x_n)$ be the function taking a vector to the parallel unit vector:
			\[u(\mathbf{x}) = \frac{\mathbf{x}}{\left \lvert { \mathbf{x} } \right \lvert }.\]Define $\varphi$ as follows:
			\[\varphi(\langle x_1, \dots x_n, x_{n+1}\rangle = \begin{cases}
						\frac{1}{2}\langle x_1, \dots x_n \rangle & 1 \leq x_{n+1} \leq 0\\
						u(x_1, \dots x_n) - \frac{1}{2}\langle x_1, \dots x_n\rangle & 0 < x_{n+1} < 1\\
						[\partial D^{n}] & x_{n+1} = 1

			\end{cases}\]
			Where $[\partial D^n]$ is the equivalence class of points in $\partial D^{n}$ under the relation $\sim$.
			\par This function is clearly continuous in the regions $1 \leq x_{n+1} < 0$ and $0 < x_{n+1} < 1$, as on each region it is defined by a single continuous function. 
			\par On the boundary $x_{n+1} = 0$, the two functions $x \mapsto \frac{1}{2} \langle x_1, \dots x_n\rangle$ and $x \mapsto u(x_1, \dots x_n) - \frac{1}{2} \langle x_1, \dots x_n\rangle$ agree, as we can see by computation. Letting $\mathbf{x} = \langle x_1, \dots , x_n\rangle$ and using the fact that $x_{n+1} = 0$ implies $\left \lvert { \langle x_1, \dots x_n\rangle  } \right \lvert  = 1$:
			\begin{align*}
				\frac{ 1}{2}\langle x_1, \dots x_n \rangle &= \langle \frac{x_1}{ 2} , \dots \frac{x_n}{2}\rangle\\
				&= \langle \frac{ 2x_1 - x_1}{2}, \dots, \frac{2x_n - x_n}{2}\rangle\\
				&= \langle \frac{x_1}{1}, \dots , \frac{ x_n}{1}\rangle - \langle \frac{x_1}{2}, \dots , \frac{x_n}{2}\rangle\\
			&= \langle \frac{x_1}{\left \lvert { \mathbf{x} } \right \lvert }, \dots , \frac{x_n}{\left \lvert { \mathbf{x} } \right \lvert }\rangle - \frac{1}{2} \langle x_1, \dots , x_n\rangle\\
			&= u(x_1, \dots , x_n) - \frac{ 1}{2} \langle x_1, \dots x_n\rangle
			\end{align*}

			Since the two functions approach the same value at the boundary between the regions, $x_{n+1} = 0$, the function $\varphi$ is continuous there. We finally see that it is continuous near $x_{n+1} = 1$. We show that, for every convergent sequence of points $\{\mathbf{x}_i\} \in S^{n} $ which approach $\langle 0, \dots , 0, 1\rangle$, the images $\varphi(\mathbf{x}_i) $ approach $[\partial D^n]$.
			\par Let $\left\{ \mathbf{x}_i \right\}$ be a sequence of points in $S^n$ that approach $\langle 0, \dots, 0, 1\rangle$. We want to show that, for every open neighborhood $U$ of $[\partial D^n]$, there is some $N \in \N$ such that, for all $m \geq N$, the images $\varphi(\mathbf{x}_m)$ all lie in $U$.
			\par Every open neighborhood $U$ of $[\partial D^n]$ is the image of some open set in $D^n$ which contains the entire preimage of $[\partial D^n]$: that is, it is an open set in $D^n$ which contains the entire boundary $\partial D^n$. This implies that it contains some open annulus of inner radius $1 - \varepsilon$ and outer radius $1$: if it did not, it would contain points arbitrarily close to the boundary $\partial D^n$ which did not lie in $U$, contradicting openness. 
			\par Because the points $\mathbf{x}_i$ approach $\langle 0, \dots , 0,1\rangle$, there must be some $N \in \N$ such that, for all $m > \N$, $\left \lvert { \langle 0, \dots , 0, 1\rangle - \mathbf{x}_m } \right \lvert  < 2\varepsilon$. This in turn implies that $\lvert\langle x_{m1}, \dots x_{mn}\rangle \rvert < 2\varepsilon$:
			\begin{align*}
				\left \lvert { \langle x_{m1}, \dots , x_{mn}\rangle} \right \lvert &= \left \lvert { \langle x_{m1}, \dots , x_{mn}, 0\rangle } \right \lvert \\
				\leq \left \lvert { \langle x_{m1}, \dots , x_{mn}, (1 - x_{m(n+1)})\rangle } \right \lvert \\
				&= \left \lvert { \langle 0, \dots , 0 , 1\rangle - \langle x_{m1}, \dots x_{m(n+1)}\rangle } \right \lvert \\
				&= \left \lvert { \langle 0, \dots , 0, 1\rangle - \mathbf{x_m }} \right \lvert \\
				&< 2\varepsilon
			\end{align*}
			Now, if $\left \lvert { \langle x_{m1}, \dots , x_{mn}\rangle } \right \lvert < 2\varepsilon$, and $x_{m(n+1)}$ is close to $1$ (and therefore greater than 0) then its image under $\varphi$ must lie either in the annulus $1 - \varepsilon < \left \lvert { x } \right \lvert  < 1$, or at the point $[\partial D^n]$. If $x_{m(n+1)} < 1$:
			\begin{align*}
				\left \lvert { \varphi(\mathbf{x}_m) } \right \lvert &= \left \lvert { u( \langle x_{m1}, \dots , x_{mn}\rangle) - \frac{1}{2} \langle x_{m1}, \dots , x_{mn}\rangle } \right \lvert \\
				&= 1 - \frac{1}{2} \left \lvert { x_{m1}, \dots , x_{mn} } \right \lvert \\
				&> 1 - \frac{1}{2}(2\varepsilon)\\
				&> 1 - \varepsilon
			\end{align*}
			And, if $x_{m+1} = 1$, the image of $\varphi(\mathbf{x}_m)$ is $[\partial D^n]$ by definition. Therefore, for $m > N$, the image of $\varphi(\mathbf{x}_m)$ lies in the union of $[\partial D^n]$ with the open annulus $1 - \varepsilon < \left \lvert {  x } \right \lvert  < 1$, which lies in $U$. So, the map $\varphi$ is continuous at $x_{n+1} = 1$.
			\par We now construct a continuous map $\psi: D^n / \sim \to S^{n}$. Let $\mathbf{x} \in D^{n}$, and define $\psi(\mathbf{x})$ as follows:
			\[ \psi(\mathbf{x}) = \begin{cases}
					\langle 2x_1, \dots , 2x_n,  -(1 - \left \lvert { 2\mathbf{x} } \right \lvert^2)^{1/2} \rangle  & \left \lvert { \mathbf{x} } \right \lvert \leq \frac{1}{2}\\
					\langle (\frac{1}{2} - \frac{1}{2}\left \lvert { \mathbf{x} } \right \lvert )x_1, \dots , (\frac{1}{2} - \frac{1}{2} \left \lvert { \mathbf{x} } \right \lvert )x_n, (1 - \left \lvert { (\frac{1}{2} - \frac{1}{2}\left \lvert { \mathbf{x} } \right \lvert} )\mathbf{x}\right \lvert^2 )^{1/2}\rangle & \frac{1}{2} < \left \lvert { \mathbf{x} } \right \lvert < 1\\
					\langle 0, \dots , 0 , 1\rangle & \mathbf x = [\partial D^n]
		\end{cases}\]
		First, we see that this is indeed a two-sided inverse to $\varphi$: on the three domains in $D^n / \sim$. If $\left \lvert { \mathbf{x} } \right \lvert \leq \frac{1}{2}$, then $-(1 - \left \lvert { 2\mathbf{x} } \right \lvert ^2)^{1/2}\leq 0$:
		\begin{align*}
			\varphi(\psi(\mathbf(x))) &= \varphi( \langle 2x_1, \dots , 2x_n, -(1 - \left \lvert {  2\mathbf{x} } \right \lvert ^2)^{1/2}\rangle)\\
			&= \frac{1}{2} \langle 2x_1, \dots , 2x_n\rangle\\
			&= \mathbf{x}
		\end{align*}
		If $\frac{ 1}{2} < \left \lvert { \mathbf{x} } \right \lvert < 1$, then $(1 - \left \lvert { (\frac{1}{2} - \frac{1}{2}\left \lvert { \mathbf{x} } \right \lvert )\mathbf{x}\right \lvert^2 )^{1/2} > 0$ :
		\begin{align*}
			\varphi(\psi(x)) &= \varphi( \langle (\frac{1}{2} - \frac{1}{2} \left \lvert { \mathbf{x} } \right \lvert ) x_1, \dots , (\frac{1}{2 } - \frac{1}{2}\left \lvert { \mathbf{x} } \right \lvert ) x_n, (1 - \left \lvert { (\frac{1}{2} - \frac{1}{2}\left \lvert { \mathbf{x} } \right \lvert )\mathbf{x} } \right \lvert ^2 )^{1/2}\rangle)\\
			&= \frac{\mathbf{x}}{\left \lvert { \mathbf{x} } \right \lvert } - 2 \left \langle \left( \frac{1}{2} - \frac{1}{2} \mathbf{x} \right)x_1, \dots , \left( \frac{1}{2} - \frac{1}{2}\left \lvert { \mathbf{x} } \right \lvert  \right)x_n\right \rangle\\
			&= \mathbf{x}
		\end{align*}
		And, finally, $\psi$ and $\varphi$ map the singletons $\langle 0, \dots , 0, 1\rangle$ and $[\partial D^n]$ to each other.
		$\psi$ is continuous as well; a full proof of this would also need to analyze the behavior of the function at the circle $\left \lvert { x } \right \lvert  = \frac{1}{2}$, and at the boundary point $[\partial D^n]$.
		\end{proof} 
	\end{problem}
	\begin{problem}{2}
		Write down an example of a topological space $X$ and an equivalence relation $\sim$ on $X$ such that the standard map to the quotient topological space $X \to X / \sim$ is not an open map.
		\begin{proof}
			The following example was found on math.stackexchange:
			Let $X = \R$, the set of real numbers with its standard topology, and let $\sim$ be the equivalence relation
			\[x \sim y \iff x = y \text{ or } (x \in \Z \text{ and }y \in \Z).\]
			Then the preimage of the image of any open set containing an integer, say $(-\varepsilon, \varepsilon)$, contains all of $\Z$:
			\[\pi^{-1}(\pi( (-\varepsilon, \varepsilon))) = (- \varepsilon, \varepsilon) \cup \Z,\]
			which is not in general open in $\R$. Thus the set $\pi( (-\varepsilon, \varepsilon))$ is not open in the quotient topology.
		\end{proof}
	\end{problem}
\end{section}
\maketitle
\end{document}
