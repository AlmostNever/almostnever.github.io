% --------------------------------------------------------------
% Andrew Tindall
% --------------------------------------------------------------
 
\documentclass[12pt]{article}
 
\usepackage[margin=1in]{geometry} 
\usepackage{amsmath,amsthm,amssymb,enumitem,hyperref,tikz-cd}

\newcommand{\N}{\mathbb{N}}
\newcommand{\Q}{\mathbb{Q}}
\newcommand{\Z}{\mathbb{Z}}
\newcommand{\C}{\mathbb{C}}
\newcommand{\R}{\mathbb{R}}
\newcommand{\mc}[1]{\mathcal{#1}}
\newcommand{\e}{\varepsilon}
\newcommand{\bs}{\backslash}
\newcommand{\PGL}{\text{PGL}}
\newcommand{\Sp}{\text{Sp}}
\newcommand{\tr}{\text{tr}}
\newcommand{\Lie}{\text{Lie}}
\newcommand{\rec}[1]{\frac{1}{#1}}
\newcommand{\toinf}{\rightarrow \infty}


\theoremstyle{definition}
\newtheorem{proofpart}{Part}
\newtheorem{theorem}{Theorem}
\makeatletter
\@addtoreset{proofpart}{theorem}
\makeatother


\newenvironment{problem}[2][Problem]{\begin{trivlist}
\item[\hskip \labelsep {\bfseries #1}\hskip \labelsep {\bfseries #2.}]}{\end{trivlist}}
 
\begin{document}
 
%\renewcommand{\qedsymbol}{\filledbox}
 
\title{Homework 11}
\author{Andrew Tindall \\
Topology 2}
\maketitle
\begin{problem}{1}
	Let $k \in \Z_{\geq 2}$ and let $X$ be a path-connexted $k$-dimensional manifold. Let $x \in X$. For what $n \in \Z_{\geq 0}$ is the natural map $H_n(H \backslash \left\{ x \right\}) \to H_n(X)$ neceessarily an isomorphism? Your answer should discuss \textit{all} $n \in \Z_{\geq 0}$.
	\begin{proof}
		\par Let $1 \leq n \leq k - 1$. Because $X$ is a $k$-dimensional manifold, the point $x$ lies in a neighborhood $A$ that is isomorphic to $\R^k$. By the excision theorem, the inclusion map $(X \backslash \left\{ x \right\}, A \backslash \left\{ x \right\}) \hookrightarrow (X, A)$ induces an isomorphism $H_n(X,A) \simeq H_n(X \backslash \left\{ x \right\}, A \backslash \left\{ x \right\})$. Writing $\tilde X = X \backslash \left\{ x \right\}$, and $\tilde A = A \backslash \left\{ x \right\}$, we have $H_n(X, A) \simeq H_n(\tilde X, \tilde A)$.
		\par As shown in Hatcher, the short exact sequence of the singular chain complexes \[0 \to C_\bullet(A) \to C_\bullet(X) \to C_\bullet(X, A)\]
		induces a long exact sequence of homology groups
		\[ \cdots \to H_n(A) \to H_n(X) \to H_n(X, A) \to H_{n-1}(A) \to \cdots.\]
		Further, this exact sequence is natural, meaning that the map $\iota: (X\backslash \left\{ x \right\}, A \backslash \left\{ x \right\}) \to (X, A)$ induces a commutative diagram
		\[\begin{tikzcd}\cdots \arrow[r] & H_n(A \backslash \left\{ x \right\}) \arrow[d, "\iota_*"] \arrow [r] & H_n(X \backslash \left\{ x \right\}) \arrow[d, "\iota_*"] \arrow[r] & H_n(X \backslash \left\{ x \right\}, A \backslash \left\{ x \right\}) \arrow[d, "\iota_*"] \arrow[r] & H_{n-1}(A \backslash \left\{ x \right\}) \arrow[d, "\iota_*"] \arrow[r] & \cdots\\
		\cdots \arrow[r] &H_n(A) \arrow[r] &H_n(X) \arrow[r] & H_n(X,A) \arrow[r] & H_{n-1}(A) \arrow[r] & \cdots \end{tikzcd}\]
		Now, by the fact that $1 \leq n \leq k-1$, and $A \simeq \R^k$, both groups $H_n(A \backslash \left\{ x \right\})$ and $H_n(A)$ are trivial, and the groups $H_{n-1}(A\backslash \left\{ x \right\} )$ and $H_{n-1}(A)$ are either both $\Z$, if $n = 1$, or $0$ otherwise, and either way the induced map $\iota_*$ is an isomorphism. Finally, by excision, the map $\iota_*: H_n(X \backslash \left\{ x \right\}, A \backslash \left\{ x \right\}) \to H_n(X, A)$ is an isomorphism for each $n$. So, we have the following diagram, with exact rows:
		\[\begin{tikzcd} H_{n+1}(X, A) \arrow[d, "\iota_*"]\arrow[r] & 0 \arrow[d] \arrow [r] & H_n(X \backslash \left\{ x \right\}) \arrow[d, "\iota_*"] \arrow[r] & H_n(X, A) \arrow[d, "\iota_*"] \arrow[r] & H_{n-1}(A \backslash \left\{ x \right\}) \arrow[d, "\iota_*"]\\
		H_{n+1}(X,A) \arrow[r] &H_n(A) \arrow[r] &H_n(X) \arrow[r] & H_n(X,A) \arrow[r] & H_{n-1}(A) \arrow[r] & \cdots \end{tikzcd}\]

	\end{proof}
\end{problem}
\end{document}
