% --------------------------------------------------------------
% Andrew Tindall
% --------------------------------------------------------------
 
\documentclass[12pt]{article}
 
\usepackage[margin=1in]{geometry} 
\usepackage{amsmath,amsthm,amssymb,enumitem}
\setlist{
	listparindent=\parindent,
parsep=0pt,}

\newcommand{\N}{\mathbb{N}}
\newcommand{\Q}{\mathbb{Q}}
\newcommand{\Z}{\mathbb{Z}}
\newcommand{\R}{\mathbb{R}}
\newcommand{\mc}[1]{\mathcal{#1}}
\newcommand{\e}{\varepsilon}
\newcommand{\bs}{\backslash}
\newcommand{\PGL}{\text{PGL}}
\newcommand{\Sp}{\text{Sp}}
\newcommand{\tr}{\text{tr}}
\newcommand{\Lie}{\text{Lie}}
\newcommand{\rec}[1]{\frac{1}{#1}}
\newcommand{\toinf}{\rightarrow \infty}


\theoremstyle{definition}
\newtheorem{proofpart}{Part}
\newtheorem{theorem}{Theorem}
\makeatletter
\@addtoreset{proofpart}{theorem}
\makeatother


\newenvironment{problem}[2][Problem]{\begin{trivlist}
\item[\hskip \labelsep {\bfseries #1}\hskip \labelsep {\bfseries #2.}]}{\end{trivlist}}
 
\begin{document}
 
%\renewcommand{\qedsymbol}{\filledbox}
 
\title{Homework 4}
\author{Andrew Tindall\\
	Algebra II}
 
\maketitle
\begin{section}{Problems}
\begin{problem}{1}
Dummit \& Foote Problem 7.5.2: Let $R$ be an integral domain and let $D$ be a nonempty, multiplicatively closed subset of $R$. Prove that the ring of fractions $D^{-1}R$ is isomorphic to a subring of the field of fractions of $R$.
\end{problem}
\begin{proof}
Let $F_R$ be the field of fractions of $R$. There are two canonical maps, $j : R \to D^{-1}R$ and $i : R \to F_R$, both of which send $r \in R$ to the formal fraction $\frac{r}{1}$. 
\par If $D$ contains zero, then $D^{-1}R = 0$, which is trivially isomorphic to the subring $0$ of $F_R$. So, we can assume that $D$ does not contain zero, the only zerodivisor of $R$. Therefore, every element of $D \subset R$ is mapped to a unit in $R_F$ under the map $i$, and we can invoke the  universal property of the localization.
\par The universal property states that $j$ is initial in the subcategory of $R/\text{cRing}$ consisting of those maps taking every element of $D$ to a unit - that is, given a map $f: R \to S$ which maps $D$ to units, there is a unique map $g : D^{-1}R \to S$ such that $g \circ j = f$. 
\par In particular, the map $i : R \to F_R$ can be factored through $j$ by a unique map $g : D^{-1}R \to F_R$ such that $i = g \circ j$. We need only show that $g$ is injective, which will show that $D^{-1}R$ is ismorphic to the image of $g$, which is a subring of $F_R$. 
\par To see that this map is injective, let $\frac{r_1}{d_1}, \frac{r_2}{d_2}$ be two elements of $D^{-1}R$ such that $g(r_1/d_1) = g (r_2/d_2)$.
\par We first note that for any element $d \in D$, $g(1/d) = (g(d/1))^{-1}$, since 
\begin{align*}
    g \left (\frac{1}{d} \right )g \left (\frac{d}{1} \right) &= g \left ( \frac{1}{d} \frac{d}{1} \right )\\
    &= g(1)\\
    &= 1
\end{align*}
Using this, we make the following calculation:
\begin{align*}
    g \left (\frac{r_1}{d_1} \right) &= g \left ( \frac{r_1}{1} \frac{1}{d_1} \right )\\
    &= g\left (\frac{r_1}{1} \right ) g \left (\frac{1}{d_1} \right) \\
    &= g(j(r_1)) g \left ( \frac{1}{d_1} \right)\\
    &= \frac{r_1}{1} \left(g \left ( \frac{d_1}{1} \right )\right)^{-1}\\
    &= \frac{r_1}{1} \frac{1}{g ( j(d_1))}\\
    &=\frac{r_1}{1} \frac{1}{d_1/1}\\
    &= \frac{r_1}{1} \frac{1}{d_1}\\
    &= \frac{r_1}{d_1}
\end{align*}
Similarly, $g(r_2/d_2) = r_2/d_2)$. So, if $g(r_1/d_1) = g(r_2/d_2)$, then it must be true that $\frac{r_1}{d_1} = \frac{r_2}{d_2}$ as elements of the field of fractions. By definition of equality in this field, there must be some nonzero $r \in R$ such that $rr_1d_2 = rr_2d_1$. Then, since $R$ is an integral domain, we can cancel $r$ to see that $r_1d_2 = r_2d_1$, and finally that $\frac{r_1}{d_1} = \frac{r_2}{d_2}$ as elements of the localization $D^{-1}R$. This is sufficient to see that the map $g$ is injective, and thus that it is an isomorphism of $D^{-1}R$ with the subring $\text{im}(g)$ of $F_R$.
\end{proof}
\begin{problem}{2}
Dummit \& Foote, 15.4.18. Prove that $R_f$, the localization of $R$ away from $f$, is isomorphic to the quotient ring $R[x](fx - 1)$ if $f$ is not nilpotent in $R$. 
\end{problem}
\begin{proof}
We define the localization $R_f$ as $S^{-1}R$, where $S$ is the multiplicative set formed by all powers $1, f, f^2, ... $ of $f$. 
\par We first construct a surjective homomorphism $\varphi$ from $R[x] \to R_f$. We extend the map taking $r \in R$ to $\frac{r}{1}$, and $x \in R$ to $\frac{1}{f}$, so that an arbitrary element $a \in R[x]$, where $a$ is some arbitray polynomial with coefficients $a_i$, is mapped as follows:
\[\varphi \left( \sum a_i x^i \right) = \sum \frac{a_i}{f^i}\]
Now, we show that every element of $(fx -1)$ is mapped to zero under this map - let $b \in (fx - 1)$, say $b = b' \cdot (fx - 1)$. Then 
\begin{align*}
	\varphi(b) &= \varphi(b' \cdot (fx - 1))\\
	&= \varphi(b') \cdot \varphi(fx - 1)\\
	&= \varphi(b') \cdot (\frac{f}{f} - 1)\\
	&= \varphi(b') \cdot (1 - 1)\\
	&= 0
\end{align*}
Therefore, $\varphi$ can be lowered to a surjective homomorphism from $R[x]/(fx - 1)$ to $S^{-1}R$. 
\par We now show that $\varphi$ admits an inverse. We can construct the inverse function by again appealing to the universal property of the localization. The canonical map $R \to R[x]$ which identifies $R$ with the scalars of $R[x]$, when composed with the quotient map $R[x] \to R[x]/(fx - 1)$, gives a function $i : R \to R[x]/(fx - 1)$ which maps every element $f^n$ of $S$ to a unit, since $x^nf^n = 1$ in this quotient ring.
\par We therefore know that there is a unique map $g : S^{-1}R \to R[x]/(fx - 1)$ such that $g \circ j = i$. In particular, $g$ maps an element $\frac{r}{1}$ to $r$, and it maps the element $\frac{1}{f}$ to $f^{-1}$, which in the ring $R[x]/(fx - 1)$ is equal to $x$. 
\par So, calculating the value of $g\circ \varphi$ on arbitrary elements of $R$ gives $g(\varphi(r)) = g(1/r) = r$, and on $x$, gives $g(\varphi(x)) = g(1/f) = x$. This determines the map $g \circ \varphi$ as the identity. Therefore, $g$ is an inverse to $\varphi$, and the two rings are isomorphic.
\end{proof}
\end{section}
\begin{section}{Extra Stuff}
	\begin{problem}{1}
		Dummit \& Foote, 15.4.2: Let $I$ be an ideal in a commutative ring $R$, let $D$ be a multiplicatively closed subset of $R$ with ring of fra ctions $S^{-1}R$, and let $^c(^eI)$ be the satiration of $I$ with respect to $S$.
		\begin{enumerate}[label=(\alph*)]
			\item Prove that $^c(^eI) = R$ if and only if $^eI = S^{-1}R$ if and only if $I \cap S \neq 0$.
			\item Prove that $I = {}^c(^eI)$ is saturated with respect to $S$ if and only if for every $s \in S$, if $sa \in I$ then $a \in I $.
			\item Prove that extension and contraction define inverse bijections between the ideals of $R$ saturated with respect to $S$ and the ideals of $S^{-1}R$.
			\item Let $I = (2x, 3y) \subset \Z[x,y]$. Show the saturation of $I$ with respect to $\Z - \left\{ 0 \right\}$ is $(x,y)$.
		\end{enumerate}
	\end{problem}
	\begin{proof}
		Writing $\pi$ for the canonical map $R \to S^{-1}R$, we note that $^cJ = \pi^{-1}J$ for any ideal $J$ of $S^{-1}R$.
		\begin{enumerate}[label=(\alph*)]
			\item $^c(^eI) = R$ if and only if $\pi^{-1}(^eR) = R$, if and only if the ideal $^eI$ contains the whole image $\pi(R)$ of the ring $R$. Since $\frac{1}{1} \in \pi(R)$, this occurs if and only if the ideal $^eI$ is the whole ring $S^{-1}R$.
				\par In turn, $^eI = S^{-1}R$ if and only if $^eI$ contains $\frac{1}{1}$. Every element of $^eI$ may be written as $\frac{i}{s}$ for some elements $i \in I$, $s \in S$, so $\frac{1}{1} \in {^eI}$ if and only if $\frac{i}{s} = 1$ for some $i, s$, which occurs if and only if some $s \in S$ is also in $I$, i.e. iff they have nonempty intersection.
			\item $I$ is saturated with respect to $S$ iff $I = \pi^{-1}(^eI)$. One inclusion $I \subset \pi^{-1}(^eI)$ is immediate, since each $i \in I$ is the inverse image of $\frac{i}{1} \in {^eI}$. So we show that the reverse inclusion $\pi^{-1}(^eI) \subset I$ holds if and only if, for every $ s \in S$, if $sa \in I$ then $a \in I$.
				\par Assume first that the condition holds, and let $a \in \pi^{-1}(^eI)$ be arbitrary. Then $\pi(a) = \frac{a}{1}$ may be written as $\frac{i}{s}$ for some $s$ in $S$, $i \in I$. This means that $s'sa = s'i$ for some $s' \in S$. But since $s'i \in I$, our condition implies that $a \in I$.
				\par On the other hand, assume that $\pi^{-1}(^eI) \subset I$, and let $s \in S$, $a \in R$ be arbitrary elements such that $sa \in I$. Then the element $\pi(a) = \frac{a}{1} = \frac{sa}{s} \in {}^eI$. This means that $a \in \pi^{-1}(^eI)$, which by assumption means that $a \in I$.
			\item This follows quickly from the observation that an ideal of $R$ being saturated means that $^c(^eI) = I$, and that (as shown in D\&F), for every ideal $J$ of $S^{-1}R$, $^e(^cJ) = J$. Restricted to these domains, $^c$ and $^e$ are inverses, and therefore form a bijection.
			\item We show first that any element $a$ of $(x,y)$ may be written as $\frac{a}{1 } = \frac{i}{z}$, where $i \in (2x, 3y) $ and $z \in \Z$. Let $\sum_{i + j \geq 1} a_{ij}x^iy^j$ be an element of $(x,y)$, and let $I, J$ be the maximum values of $i$, $j$ respectively, such that $a_{ij} \neq 0$. Then 
				\begin{align*} 
				\frac{a}{1} &= \frac{\sum_{i + j \geq 1}a_{ij}x^iy^j}{1} \\
			&= \frac{\sum_{i + j \geq 1}(2^I3^J)a_{ij}x^iy^j}{2^I3^J} \\
			&= \frac{\sum_{i + j \geq 1}2^{I - i}3^{J-j}a_{ij}(2x)^i(2y)^j}{2^I3^J}  \in {^e}I
		\end{align*}
		We now show that no element $r$ of $\Z[x,y] \bs (x,y)$ may be written as $\frac{r}{1} = \frac{i}{s}$ for $i\in I$, $s \in \Z$. Since $\Z[x,y]\bs (x,y)$ is simply the scalars $\Z$, this is an argument by minimal degree - this can only occur if $ss'r = s'i$ for nonzero $s' \in \Z$. The degree of $i$ is at least $1$, $s'$ is nonzero, and $\Z$ is an integral domain, so the degree of $s'r$ is at least $1$. Therefore the degree of $ss'r $ is at least $1$, and because $s$ and $s'$ are both integers, the degree of $r$ is at least $1$ and it cannot be a scalar.
		\end{enumerate}
	\end{proof}
	\begin{problem}{2}
		Dummit \& Foote, 7.5.5: If $F$ is a field, prove that the field of fractions of $F[ [x]]$ is the ring $F( (x))$ of formal Laurent series. Show the field of fractions of the ring $\Z[ [x]]$ is \textit{properly} contained in the field of Laurent series $\Q( (x))$. 
	\end{problem}
	\begin{proof}
		We construct a homomorphism $\varphi : F[ [x]] \to F( (x))$, and show that it is both injective and surjective. Let $g, h \in F[ [x]]$ be formal power series, with coefficients $g_n$ and $h_n$, and let $h_n$ be nonzero:
		\begin{align*}
			g &= \sum_n g_nx^n\\
			h &= \sum_n h_nx^n
		\end{align*}
		The element $g/h$ is a generic element of the field of fractions of $F[ [x]]$. We want $\varphi(g/h)$ to be ``$g/h$'' in some reasonable way. It is possible to divide formal power series using the formula for $1/h$, which gives a well-defined power series as long as the zero-th coefficient $h_0$ is nonzero (D\&F, Exercise 7.2.3). This is not necessarily true for our $h$, but it has at least one nonzero coefficient; let $h = x^{i}h'$, where $i$ is the degree of the lowest nonzero term of $h$. Then $h^{-1} = x^{-i}h'^{-1}$, which is a well-defined formal Laurent series.
		\par We define $\varphi$'s value on $g/h$ as follows:
		\[\varphi(g/h) = x^{-i}\cdot g\cdot h'^{-1} = gh^{-1}\]
		\par This is a well-defined function; if $g_1/h_1 = g_2/h_2$, then $g_1h_2 = g_2h_1$, so
		\begin{align*}
			\varphi(g_1/h_1) &= g_1 \cdot h_1^{-1}\\
			&= (h_2^{-1}h_2)\cdot g_1 \cdot h_1^{-1} \cdot (g_2g_2^{-1})^{-1}\\
			&= h_2^{-1} \cdot (h_1 g_2) \cdot  (h_1 g_2)^{-1} g_2\\
		&= h_2^{-1} \cdot g_2\\
		&= \varphi(g_2/h_2)
		\end{align*}
		It is also a ring homomorphism: $\varphi$ takes $1$ to $1$, scalars factor out of the denominator, and $\varphi(g_1/h_1 + g_2/h_2) = \varphi(g_1/h_1) = \varphi(g_2/h_2)$ - we show this last one with a quick calculation:
		\begin{align*}
			\varphi\left( \frac{g_1}{h_1} + \frac{g_2}{h_2} \right) &= \varphi\left( \frac{g_1h_2 + g_2h_1}{h_1h_2} \right)\\
			&= (g_1h_2 + g_2h_1) \cdot (h_1h_2)^{-1}\\
			&= g_1h_2 \cdot (h_1h_2)^{-1}  +  g_2h_1 \cdot (h_1h_2)^{-1}\\
			&= g_1h_1^{-1} + g_2h_2^{-1}\\
			&= \varphi\left( \frac{g_1}{h_1} \right) + \varphi\left( \frac{g_2}{h_2} \right)
		\end{align*}
		\par We can also see that $\varphi$ is injective, but I won't run through the proof here; we finally see that it is surjective, as if $\sum_{i=-n}^{\infty}g_ix^i$ is a formal Laurent series, then $g = x^{-n}g'$, where $g'$ is a formal power series with no terms with negative exponents, and $g = \varphi(g'/x^i)$.
		\par It is \textit{not} true that, for a more general ring, the ring of fractions of its polynomial ring is equal to the ring of Laurent series over that ring's field of fractions: for example, the field of fractions of $\Z[ [x]]$ does not contain $\Q( (x))$: the series $\sum_{n\geq 0} \frac{x^n}{n!}$ is not equal to the formal fraction of any two power series with coefficients in $\Z$, as the denominators of its terms grow too quickly (?)
	\end{proof}
	\begin{problem}{3}
	Dummit \& Foote, 7.4.30: Let $I$ be an ideal of the commutative ring $R$. Prove that the radical of $I$ is an ideal containing $I$, and that $(\text{rad }I)/I = \mathfrak N (R/I)$, the nilradical of $R/I$. \end{problem}
	\begin{proof}
		It is clear that the radical of $I$ contains $I$, since for any $x \in I$, $x^1 \in I$. On the other hand, if $f$ and $g$ are members of the radical of $I$, say $f^n \in I$ and $g^m \in I$, then $f+g$ is also a member of the radical of $I$, because $(f + g)^{m+n} \in I$, because every term of $(f+g)^{m+n}$ either has $f^n$ or $g^m$ as a factor, and if $f$ is a member of the radical of $I$ with $f^n \in I$, and $a\in R$ is an arbitrary element, then $(fa)^n = f^na^n \in R$. Therefore $\text{rad }I$ is an ideal.
		\par We now show that $(\text{rad }I)/I = \mathfrak{N}(R/I)$. If $f + I \in (\text{rad }I)/I$, then $(f + I)^n = f^n + I = 0$, so $f+I$ is in the nilradical of $R/I$. On the other hand, if $f+I$ is in the nilradical, then $(f + I)^n = f^n + I = 0$, meaning that $f^n \in I$ for some $n$, so $f + I$ is in $\text{rad}(I) + I$. So indeed the two ideals are equal.
	\end{proof}
\end{section}

\end{document}
