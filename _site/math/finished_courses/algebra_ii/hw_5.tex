% --------------------------------------------------------------
% Andrew Tindall
% --------------------------------------------------------------
 
\documentclass[12pt]{article}
 
\usepackage[margin=1in]{geometry} 
\usepackage{amsmath,amsthm,amssymb,enumitem}

\newcommand{\N}{\mathbb{N}}
\newcommand{\Q}{\mathbb{Q}}
\newcommand{\Z}{\mathbb{Z}}
\newcommand{\R}{\mathbb{R}}
\newcommand{\mc}[1]{\mathcal{#1}}
\newcommand{\e}{\varepsilon}
\newcommand{\bs}{\backslash}
\newcommand{\PGL}{\text{PGL}}
\newcommand{\Sp}{\text{Sp}}
\newcommand{\tr}{\text{tr}}
\newcommand{\Lie}{\text{Lie}}
\newcommand{\rec}[1]{\frac{1}{#1}}
\newcommand{\toinf}{\rightarrow \infty}


\theoremstyle{definition}
\newtheorem{proofpart}{Part}
\newtheorem{theorem}{Theorem}
\makeatletter
\@addtoreset{proofpart}{theorem}
\makeatother


\newenvironment{problem}[2][Problem]{\begin{trivlist}
\item[\hskip \labelsep {\bfseries #1}\hskip \labelsep {\bfseries #2.}]}{\end{trivlist}}
 
\begin{document}
 
%\renewcommand{\qedsymbol}{\filledbox}
 
\title{Homework 5}
\author{Andrew Tindall\\ Algebra II}
 
\maketitle
\begin{problem}{1}
	Dummit \& Foote 15.1.3: Prove that the field $k(x)$ of rational functions over $k$ in the variable $x$ is not a finitely generated $k$-algebra.
\end{problem}
\begin{proof}
	The following proof is is adapted from a stackexchange comment at \cite{se}.
	\par Say $k(x)$ were generated by finitely many rational functions $f_1, \dots f_n$. Any element of the $k$-algebra generated by these elements may be written as a polynomial in $f_1, \dots f_n$, with coefficients in $k$. Say $y$ is such an element. Then $y$ may also be written as a fraction $f(x)/(g(x))^n$, where $f$ is some polynomial, $g$ is the product of the denominators of the generators $f_i$, and $n$ is the degree of $y$. However, there are infinitely many irreducibles in $k[x]$, and one of them, say $p(x)$, does not divide $g(x)$. Then the element $1/p(x)$ may not be written as a ratio $f/g^n$, and is therefore not in the algebra generated by $f_1, \dots f_n$. Thus we cannot find a finite set of generators of $k(x)$.
\end{proof}
\begin{problem}{2}
	Dummit \& Foote, 15.1.6: Suppose that $0 \to M' \to M \to M'' \to 0$ is an exact sequence of $R$-modules. Prove that $M$ is a Noetherian $R$-module if and only if $M'$ and $M''$ are Noetherian $R$-modules.
\end{problem}
\begin{proof}
	\begin{itemize}
		\item $M$ Noetherian $\Rightarrow$ $M''$ Noetherian: let $I_1 \subset I_2 \subset I_3 \dots$ be an increasing chain of submodules in $M''$. Then their preimages in $M$ are an increasing chain of submodules, and by surjectivity of the map $M \to M''$, the preimages are distinct if the submodules in $M''$ are distinct. The chain in $M$ must stabilize by Noetherianness; so therefore so must the chain in $M''$.
		\item $M$ Noetherian $\Rightarrow$ $M'$ Noetherian: let $J_1 \subset J_2 \subset J_3 \dots$ be ain increasing chain of submodules in $M'$. Their images in $M$ also form an increasing chain of submodules, which eventually stabilizes; because the map $M' \to M$ is injective, the equality of the image of two submodules means that those two submodules are equal, and thus that the chain in $M'$ terminates as well. Therefore $M'$ is Noetherian.
		\item $M'$ and $M''$ Noetherian $\Rightarrow$ $M$ Noetherian: Let $L_1 \subset L_2 \subset L_3 \dots$ be an increasing chain of submodules in $M$. Then the inverse images of each submodule forms a chain in $M'$, and the images of each form a chain in $M''$, both of which eventually terminate by Noetherianness. The conclusion that the chain $L_i$ itself terminates follows from the following bit of diagram chasing:
			\par Let $\alpha: M' \to M$ be injective and $\beta: M \to M''$ be surjective, with the image of $\alpha$ equal to the kernel of $\beta$. Let $L_1 \subset L_2$ be two submodules of $M$ such that $\alpha^{-1}(L_1) = \alpha^{-1}(L_2)$, and $\beta(L_1) = \beta(L_2)$. Then $L_1 = L_2$.
			\par It suffices to show that $L_2 \subset L_1$. Let $l \in L_2$. Then $\beta(l) \in \beta(L_2) = \beta(L_1)$, so there exists some $l' \in L_1$ such that $\beta(l) = \beta(l')$. Therefore $\beta(l - l') = 0$, so $l - l'$ is in the kernel of $\beta$, which equals the image of $\alpha$. Thus $l - l' = \alpha(m)$ for some element $m \in M'$. Now, because $L_1 \subset L_2$, both $l$ and $l'$ are in $L_2$, meaning that $l-l' \in L_2$, so $m \in \alpha^{-1}(L_2) = \alpha^{-1}(L_1)$. Therefore $\alpha(m) = l-l' \in L_1$. Because $l' \in L_1$, we see that $l$ is in $L_1$, which was to be shown. 
			\par Thus, any chain of submodules $L_1 \subset L_2 \subset L_3 \dots$ such that $\alpha^{-1}(L_i) = \alpha^{-1}(L_{i+1})$ and $\beta(L_i) = \beta(L_{i+1})$ must stabilize, meaning $M$ is Noetherian.
	\end{itemize}
\end{proof}
\begin{thebibliography}{}
	\bibitem{se}{Prove that the field $k(x)$ of rational functions over $k$ in the variable $x$ is not a finitely generated $k$-algebra., URL (version: 2018-01-17): https://math.stackexchange.com/q/2608744}
\end{thebibliography}
\end{document}
