% --------------------------------------------------------------
% Andrew Tindall
% --------------------------------------------------------------
 
\documentclass[12pt]{article}
 
\usepackage[margin=1in]{geometry} 
\usepackage{amsmath,amsthm,amssymb,enumitem,hyperref,tikz-cd}

\newcommand{\N}{\mathbb{N}}
\newcommand{\Q}{\mathbb{Q}}
\newcommand{\Z}{\mathbb{Z}}
\newcommand{\C}{\mathbb{C}}
\newcommand{\R}{\mathbb{R}}
\newcommand{\mc}[1]{\mathcal{#1}}
\newcommand{\e}{\varepsilon}
\newcommand{\bs}{\backslash}
\newcommand{\PGL}{\text{PGL}}
\newcommand{\Sp}{\text{Sp}}
\newcommand{\tr}{\text{tr}}
\newcommand{\Lie}{\text{Lie}}
\newcommand{\rec}[1]{\frac{1}{#1}}
\newcommand{\toinf}{\rightarrow \infty}


\theoremstyle{definition}
\newtheorem{proofpart}{Part}
\newtheorem{theorem}{Theorem}
\makeatletter
\@addtoreset{proofpart}{theorem}
\makeatother


\newenvironment{problem}[2][Problem]{\begin{trivlist}
\item[\hskip \labelsep {\bfseries #1}\hskip \labelsep {\bfseries #2.}]}{\end{trivlist}}
 
\begin{document}
 
%\renewcommand{\qedsymbol}{\filledbox}
 
\title{Final Exam}
\author{Andrew Tindall
\\ Algebra I}
 
\maketitle
\begin{problem}{1}
	\begin{enumerate}[label=(\alph*)]
		\item Let $B$ be the (Borel) group consisting of nonsingular $2 \times 2$ matrices $(a_{ij})$ with entries in $F_3 = \left\{ -1, 0,1  \right\}$ and with $a_{21} = 0$. Show that $B$ is isomorphic to $D_{12}$, the group of symmetries of a regular hexagon.
			\begin{proof}
				First, we see that there are $12$ elements of $B$. Any nonsingular $2\times 2$ upper triangular matrix must have nonzero elements on its diagonal, and clearly any upper triangular matrix with nonzero elements on its diagonal must have nonzero determinant and so is nonsingular. There are $2$ arbitrary nonzero choices each for the two diagonal elements, and the remaining nonzero element is arbitrary, so there are a total of $3$ choices for it; this gives a total of $12$ elements of the group $B$:
				\begin{align*}
					&e = \begin{bmatrix}
						1 & 0 \\ 0 & 1
					\end{bmatrix} & s = \begin{bmatrix}
						-1 & -1 \\ 0 & 1
					\end{bmatrix} \\
					& r = \begin{bmatrix}
						-1 & 1 \\ 0 & -1		
					\end{bmatrix} & rs = \begin{bmatrix}
						1 & -1 \\ 0 & -1	
					\end{bmatrix} \\
					& r^2 = \begin{bmatrix}
						1 & 1 \\ 0 & 1
					\end{bmatrix} & r^2s = \begin{bmatrix}
						-1 & 0 \\ 0 & 1
					\end{bmatrix}\\
					& r^3 = \begin{bmatrix}
						-1 & 0 \\ 0 & -1	
					\end{bmatrix} & r^3s = 
					\begin{bmatrix}
						1 & 1 \\ 0 & -1	
					\end{bmatrix}\\
					& r^4 = \begin{bmatrix}
						1 & -1 \\ 0 & 1
					\end{bmatrix} & r^4s = \begin{bmatrix}
						-1 & 1 \\ 0 & 1
					\end{bmatrix}\\
					&r^5 = \begin{bmatrix}
						-1 & -1 \\ 0 & -1
					\end{bmatrix} & r^5s = \begin{bmatrix}
						1 & 0 \\ 0 & -1
					\end{bmatrix}
				\end{align*}
				Indeed, this exhausts the list of elements of $B$, and we see that $B$ is generated by our choice of $r$ and $s$, and further that these elements satisfy the defining equations of $D_{12}$: $r^6 = s^2 = e$, and $rs = sr^5$.
			\end{proof}
		\item Working in B, express $B$ as a semidirect product by explicitly highlighting the automorphisms acting on a normal subgroup.
			\begin{proof}
				We see that the groups generated by $r$ and by $s$ together generate $B$. Further, their intersection is trivial: the only nontrivial element in $\langle r\rangle$ whose degree divides the degree of $\langle s \rangle$, which is $2$, is $r^2 = \begin{bmatrix}
					-1 & 0 \\ 0 & -1
				\end{bmatrix}$, which is not equal to the only nontrivial element of $s$, which is $s$ itself, $\begin{bmatrix}
					-1 & -1 \\ 0 & 1
				\end{bmatrix}$. Finally, $\langle r\rangle$ is normal, since it has index $2$. So, the group $B$ must be expressable as some semidirect product $\langle r\rangle \rtimes \langle s\rangle$. We also know that $\langle s\rangle$ is not normal, since for example $rsr^{-1} = \begin{bmatrix}
					-1 & 0 \\ 0 & 1
				\end{bmatrix} \notin \left\{ e, s \right\}$. So, the semidirect product must be generated by a nontrivial homomorphism $\langle s\rangle \to \text{Aut}(\langle r\rangle)$.
				\par Since $\langle r\rangle$ is cyclic of order $6$, its automorphism group is isomorphic to $\text{Aut}(\Z_6) \simeq \text{Aut}(\Z_2 \times \Z_3)$. Any automorphism of $\Z_2 \times \Z_3$ must fix the unique element of order $2$, so it can only act nontrivially on $\Z_3$: $\text{Aut}(\Z_6) \leq \text{Aut}(\Z_3) \simeq \Z_2$, and there can only be one nontrivial automorphism of $\Z_6$. In fact, there is one, which takes the generator $1$ to $-1$, or $5$; in terms of $\langle r\rangle$ this would take the generator $r $ to $r^{5}$. 
				\par Indeed, we see that $s$ acts on $\langle r\rangle$ via this automorphism: conjugation by $s$ takes $r$ to $r^5$, as we see from the equation $rs = sr^5$ seen above. So, this is the semidirect product structure $B = \langle r\rangle \rtimes \langle s\rangle$.
			\end{proof}
	\end{enumerate}
\end{problem}
\begin{problem}{2}
	Let $G$ be a finite group, $N$ a normal subgroup of $G$ and $H$ a subgroup of $G$. If $\left \lvert {  H } \right \lvert $ is coprime to the index $\left \lvert { G : N } \right \lvert $, show that $H$ is contained in $N$. If $\left \lvert { N } \right \lvert $ is coprime to the index $\left \lvert { G : H } \right \lvert $, show that $N$ is contained in $H$.
	\begin{proof}
		First, assume that $\left \lvert { H } \right \lvert $ is coprime to the index $\left \lvert { G:N } \right \lvert $. Let $\pi: G \to G/N$ be the natural quotient map, so that $\text{ker}(\pi) = N$. The order of $G/N$ must be $\left \lvert { G:N } \right \lvert$. Then $\left \lvert { \pi(H) } \right \lvert$ must divide $\left \lvert { H } \right \lvert $ and $\left \lvert { G/N } \right \lvert  = \left \lvert { G:N } \right \lvert $. Since these two numbers are coprime, $\left \lvert { \pi(H) } \right \lvert  = 1$, so $\pi(H)$ must be trivial. Thus $H$ is contained in the kernel of $\pi$, which is $N$.
		\par On the other hand, assume that $\left \lvert { N } \right \lvert $ is coprime to $\left \lvert { G:H } \right \lvert $. Again, let $\pi: G \to G/N$ be the natural map. Then $\pi(H) = H / H \cap N$, and $\left \lvert { \pi(H) } \right \lvert  = \left \lvert { H } \right \lvert / \left \lvert { H \cap N } \right \lvert $.
	\end{proof}
\end{problem}
\begin{problem}{3}
	Suppose that $EF$ are finite Galois extensions of field $Q$, both contained in a common extension $L$ of $Q$. 
	\begin{enumerate}[label=(\alph*)]
		\item The compositum $EF$ is Galois over $Q$.
			\begin{proof}
				The following proof is mostly from Dummit \& Foote, p. 592.
				\par The Galois extension $E$ of the rational numbers $Q$ must be the splitting field of some separable polynomial $f(x)$ in $Q[x]$. Similarly, the extension $F$ must be the splitting field for some polynomial $g(x)$ in $Q[x]$. Then $f(x)g(x)$ is a polynomial whose roots all lie in $EF$, and $EF$ is the smallest such field, so $EF$ is the splitting field for the polynomial $f(x)g(x)$.
				\par This polynomial is not necessarily separable, but since we are working over $Q$, where every irreducible polynomial is separable, the squarefree part of $f(x)g(x)$ is a product of unique irreducibles and is also separable. So, $EF$ is the splitting field of the separable polynomial $f(x)g(x)$ in $Q[x]$, and so it is Galois over $Q$.
			\end{proof}
		\item The intersection $E \cap F$ is Galois over $Q$.
			\begin{proof}
				We want to show $E \cap F$ is Galois over $Q$; it suffices to show that any irreducible polynomial in $Q[x]$ with a root in $E \cap F$ splits completely over $E \cap F$.
				\par Let $f \in Q[x]$ be irreducible with a root $ \alpha$ in $E \cap F$. Since $\alpha \in E$, and $E$ is Galois over $Q$, we see that $f$ must split over $E$. Similarly, $\alpha \in F$, so $f$ splits over $F$. Since all the roots of $f$ lie in both $F$ and $E$, they all lie in the field $F \cap E$. Since any irreducible polynomial in $Q[x]$ with a root in $F \cap E$ splits over $F \cap E$, it must be Galois over $Q$.
			\end{proof}
	\end{enumerate}
\end{problem}
\begin{problem}{4}
	\begin{enumerate}[label=(\alph*)]
		\item Which integers $d \in \Z$ satisfy $d\Z = \left\{ 34a - 51b: a, b \in \Z \right\}$?
			\begin{proof}
				Because $17$ divides both $34$ and $51$, any term $34a - 51b$ must be divisible by $17$; on the other hand, since $34 \cdot 2 - 51 = 17$, and the set $\left\{ 34a - 51b : a, b \in \Z \right\} = (34, 51)$ is an ideal of $\Z$ and therefore a ring, every multiple of $17$ is contained in this ring. So, this ring is exactly equal to the set of multiples of $17$, and the integers $d$ which satisfy $d\Z = (34, 51)$ are the associates of $17$ in $\Z$, which are simply
				\[\left\{ -17, 17 \right\}.\]
			\end{proof}
		\item If $F$ is a field, show that 
			\[L = \left\{ \frac{f}{g}: f, g \in F[X], g \neq 0 \right\}\]
			is a field extension of $F$. Use this to answer whether there are infinite fields of characteristifc $p$.
			\begin{proof}
				First, we see that $L$ is a ring with unity. It certainly contains a unit element, namely $\frac{1}{1}$. It is also closed under multiplication, since $\frac{f_1}{g_1} \cdot \frac{f_2}{g_2} = \frac{f_1f_2}{g_1g_2}$, and $g_1g_2$ is nonzero if both $g_1$ and $g_2$ are. Finally, it is closed under addition, since $\frac{f_1}{g_1} + \frac{f_2}{g_2} = \frac{f_1g_2 + f_2g_1}{g_1g_2}$.
			Next, it is clear that $F$ is a subring of $L$, since for any $f \in F$, mapping $f$ to the element $\frac{f}{1}$ gives a ring homomorphism from $F$ to $L$ - it is a homomorphism, since $\frac{f_1}{1} \cdot \frac{f_2}{1} = \frac{f_1f_2}{1}$, and $\frac{f_1}{1} + \frac{f_2}{1} = \frac{f_1 + f_2}{1}$.
			Finally, $L$ is a field, since for any nonzero $\frac{f}{g}$, it must be true that $f$ is nonzero as well, so the element $\frac{g}{f}$ is a well-defined element of $L$. This is an inverse to $\frac{f}{g}$, since $\frac{f}{g}\cdot \frac{g}{f} = \frac{fg}{fg} = \frac{1}{1}$. Any field $L$ with a ring homomorphism $F \to L$ is a field extension of $F$, so indeed $F \leq L$.
			\par This implies that there are infinite fields of characteristic $p$, for any prime $p$. Fixing some $p$, the field $F_p$ is contained in a field $L_p$, consisting of $\frac{f}{g}$ for $f, g \in F_p[X]$, and $g$ nonzero. But there are an infinite number of polynomials in $F_p[X]$, so the set of terms $\frac{f}{1}$, for arbitrary $f$, is infinite - hence $L_p$ is an infinite field extension of $F_p$. Extensions have the same characteristic as their base field, so $L_p$ is an infinite field of characteristic $p$.
			\end{proof}
		\item Find an irreducible polynomial of degree $100$ over $Q$ whose Galois group is cyclic.
			\begin{proof}
				Since the extension of $Q$ by the primitive $n$th roots of unity, for any $n$, has cyclic Galois group, we need only find an irreducible polynomial $f$ such that $Q[x](f(x)) = Q(\zeta)$, where $\zeta$ is some primitive root of unity.
				\par The product $\prod_{i} (x - \zeta_i) = x^n - 1$, where $\{\zeta_i\}$  is the collection of all $n$th roots of unity, is not irreducible over $Q$. However, it has at most $2$ roots in $Q$, so by simply dividing by these roots we will obtain an irreducible polynomial whose roots are exactly the non-rational $n$th roots of unity. If $n$ is even, then both $1$ and $-1$ are roots of the polynomial, while if $n$ is odd, then only $1$ is. So, if $n$ is, say, $101$, then the polynomial
				\[f(x) = \frac{\prod_{i = 1}^{101} (x - \zeta_i)}{x - 1} = x^{101} + x^{100} + \cdots + x + 1 \]
				where $\zeta_i$ are the $101$st roots of unity, is irreducible. Since $Q$ has characteristic $0$ it is also separable, and the field $Q[x]/(f(x))$ is exactly the extension by $Q$ of all of the $101$st roots of unity. This field is equal to $Q(\zeta)$, for some primitive $101$st root of unity $\zeta$; for example $\zeta = e^{2\pi i / 101}$. Therefore, it has cyclic Galois group.
			\end{proof}
		\item Find an irreducible polynomial over $Q$ that has no roots in $Q$.
			\begin{proof}
				Any polynomial with all complex roots will work, such as $f(x) = x^2 + 2$, or simply any polynomial whose roots are radicals, such as $g(x) = x^2 - 2$.
			\end{proof}
		\item Perhaps by adding in certain radicals, or otherwise, construct an algebraic extension of $Q$ that has infinite dimension over $Q$.
			\begin{proof}
				Let $\tilde Q$ be the field obtained from $Q$ by affixing $\sqrt p$ for each prime $p \in \Z$ - i.e. the colimit of the diagram
				\[ Q \to Q(\sqrt 2) \to Q(\sqrt 2, \sqrt 3) \to \cdots\]
				Which does exist, although it is not guaranteed to since \textbf{Field} is not cocomplete. Any element of $\tilde Q$ can be written as a finite sum 
				\[q_0 + q_1\sqrt {p_1} + \cdots = q_n \sqrt {p_n},\]
				For some $n \in \N$, $q_i \in Q$, and primes $p_i$. This is a field, since any two elements $a \in Q(\sqrt 2, \dots , \sqrt{p_n}) $ and $b \in Q(\sqrt 2, \dots \sqrt{p_m})$ are contained in the field extension $a \in Q(\sqrt 2, \dots , \sqrt{p_k})$, where $k = \text{max}(m, n)$. It is a field extension of $Q$, since it contains $Q$ as a subfield, and it is an algebraic extension, since any element $a \in \tilde Q$ is contained in some algebraic extension $Q(\sqrt 2, \dots \sqrt{p_m})$, and so is algebraic. Finally, it is an infinite extension of $Q$, as any finite collection of elements that generate $\tilde Q$ would be contained in a finite extension $Q(\sqrt 2, \dots , \sqrt {p_m})$ for some $m$, and the element $\sqrt{p_{m+1}}$ is not contained in this extension. 
			\end{proof}
		\item Find an irreducible cubic polynomial over $F_5$.
			\begin{proof}
				We show that the polynomial $x^3 - x + 2$ has no roots in $F_5$:
				\begin{align*}
					0^3 - 0 + 2 &= 2\\
					1^3 - 1 + 2 &= 2\\
					2^3 - 2 + 2 &= 3\\
					3^3 - 3 + 2 &= 1\\
					4^3 - 4 + 2 &= 2
				\end{align*}
				Any reducible cubic must have a linear factor, so $x^3 - x + 2$ is irreducible over $F_5$.
			\end{proof}
	\end{enumerate}
\end{problem}
\begin{problem}{5}
	Consider the element $\gamma = 2\sqrt 3 - \sqrt 2 \in L = Q(\sqrt 2, \sqrt 3)$.
	\begin{enumerate}[label=(\alph*)]
		\item Prove that $\gamma$ is a primitive element of $L$.
			\begin{proof}
				We want to show that $Q(2 \sqrt 3 = \sqrt 2) = Q(\sqrt 2, \sqrt 3)$. One direction is clear; so we want to show that $\sqrt 2 \in Q(2 \sqrt 3 - \sqrt 2)$, and similarly for $\sqrt 3$.
				\par First, we see that 
				\[\frac{1}{2 \sqrt 3 - \sqrt 2} = \frac{2 \sqrt 3 + \sqrt 2}{12 - 2} = \frac{2 \sqrt 3 + \sqrt 2}{10},\]
				so $\frac{2 \sqrt 3 + \sqrt 2}{10} \in Q(2 \sqrt 3 - \sqrt 2)$, and so is $2 \sqrt 3 + \sqrt 2$. Therefore,
				\[(2 \sqrt 3 - \sqrt 2) + (2 \sqrt 3 + \sqrt 2) = 4 \sqrt 3 \in Q(2 \sqrt 3 - \sqrt 2),\]
				So $\sqrt 3$ is in $Q(2 \sqrt 3 - \sqrt 2)$ as well. 
				\par Since $\sqrt 3 \in Q(2 \sqrt 3 - \sqrt 2)$, so is $2 \sqrt 3$, and so is $2 \sqrt 3 - (2 \sqrt 3 - \sqrt 2) = \sqrt 2$. So, the two generators $\sqrt 3$ and $\sqrt 2$ of $L$ are in $Q(2 \sqrt 3 - \sqrt 2)$, so the element generates the whole field and is primitive in $L$.
			\end{proof}
		\item Find the conjugates of $\gamma$ in $L$, first over $Q$ then over $Q(\sqrt 2)$.
			\begin{proof}
				There are four elements of the Galois group of $Q(\sqrt 2, \sqrt 3)$ over $Q$, so there will be (at most) four conjugates of $\gamma$ in this field, counting $\gamma$ itself, found by permuting $\left\{ -\sqrt 2, \sqrt 2 \right\}$, and $\left\{ - \sqrt 3, \sqrt 3 \right\}$. So, the conjugates of $\gamma$ in $L$ are
				\begin{align*}
					&2 \sqrt 3 - \sqrt 2, & 2 \sqrt 3 + \sqrt 2,\\
					&-2 \sqrt 3 - \sqrt 2, &-2 \sqrt 3 + \sqrt 2.
				\end{align*}
				Similarly, there are two element of the Galois group of $L$ over $Q(\sqrt 2)$, so there are two conjugates of $\gamma$ over this field:
				\[2 \sqrt 3 - \sqrt 2, -2 \sqrt 3 + \sqrt 2\]
			\end{proof}
		\item Find the minimum polynomial of $\gamma$ over $Q$.
			\begin{proof}
				Let $f_\gamma$ be the minimal polynomial of $\gamma$ over $Q$. Since $\gamma$ has $4$ conjugates over $Q$, the degree of $f_\gamma$ must be $4$. First, we see \[\gamma^2 = 12 - 4\sqrt 6 + 2 = 14 - 4 \sqrt 6,\] and
				\[\gamma^4 = 196 - 112 \sqrt 6 + 96 = 292 - 112 \sqrt 6..\]
				since $\gamma^2 - 14 = -4 \sqrt 6$, and $\gamma^4 - 292 = -112 \sqrt 6$, we see that $\gamma$ satisfies the equation
				\[(\gamma^4 - 292) = 28(\gamma^2 - 14),\]
				or
				\[ \gamma^4 - 28\gamma^2 + 100 = 0.\]
				so, the minimal polynomial of $\gamma$ is $x^4 - 28x^2 + 100$.
			\end{proof}
		\item find the minimum polynomial of $\gamma$ over $q(\sqrt 2)$.
			\begin{proof}
				since there are two conjugates of $\gamma$ over $q(\sqrt 2)$, the minimal polynomial of $\gamma$ over $q(\sqrt 2)$ should be degree $2$. we see that
				\begin{align*} \gamma^2 &= 12 - 4\sqrt 6 + 2\\
				&= 14 - 4 \sqrt 6\\
			&= 14 - 4 \sqrt 2 \sqrt 3\\
		&= 14 - 2\sqrt 2(2 \sqrt 3 - \sqrt 2) - 2 \sqrt 2 (\sqrt 2)\\
	&= 6 - 2 \sqrt 2 (\gamma)\end{align*}
	So,  $\gamma$ satifies the equation
	\[ \gamma^2 = 6 - 2 \sqrt 2 (\gamma),\]
	or 
	\[ \gamma^2 + 2 \sqrt 2 \gamma - 6 = 0.\]
	So, the minimal polynomial of $\gamma$ over $Q(\sqrt 2)$ is $x^2 + 2 \sqrt 2 x - 6$.
			\end{proof}
	\end{enumerate}
\end{problem}

\begin{problem}{6}
	if $f$ is an irreducible polynomial of degree $d$ over $K$ and $L$ is an extension of $K$ of degree $n$, with $n$ coprime to $d$, show (perhaps by applying the tower law) that $f$ remains irreducible over $L$.
	\begin{proof}
		Let $x \in \overline K$ be an element of the algebraic closure of $K$ which satisfies $f$. Since $L$ is a finite extension of $K$, it may be considered a subfield of $\overline K$, so that $K \leq L \leq \overline K$. 
		\par Now, consider the order $\left \lvert {  L(x) : K } \right \lvert $. By the tower law, this is equal to both
		\[ \left \lvert {  L(x) : L } \right \lvert  \cdot \left \lvert {  L : K } \right \lvert,\]
		and 
		\[\left \lvert {  L(x) : K(x) } \right \lvert  \cdot \left \lvert {  K(x) : K } \right \lvert. \]
		But the degree $\left \lvert { L(x) : K } \right \lvert $ is equal to $n$, and the degree $\left \lvert {  K(x) : K } \right \lvert $ is equal to $ d$, and the two integers $n$ and $d$ are coprime. So it must be true that $d$ divides $\left \lvert { L(x) : L } \right \lvert $. However, the minimal polynomial of $x$ in $K$ has order $d$, so the minimal polynomial of $x$ in the extension $L$ of $K$ has order at most $d$, and so the minimal polynomial of $x$ in $L$ must be $f$, the same as in $K$. Since the minimal polynomial of any element of a field is irreducible over that field, $f$ must be irreducible over $L$.
	\end{proof}
\end{problem}
\begin{problem}{7}
	construct polynomials $f$ and $g$ of degree $5$ in $\R[x]$ such that each has exactly three real roots and
	\begin{enumerate}[label=(\alph*)]
		\item $f$ is solvable by radicals
		\item $g$ is not solvable by radicals.
	\end{enumerate}
	explain in detail what theorems you use.
	\begin{proof}
		We use the following theorem from Dummit \& Foote: A polynomial $g$ is solvable by radicals if and only if its Galois group is solvable. Since $g$ is to be irreducible of degree $5$, it will have $5$ distinct roots in its splitting field $K$, and so the elements of $\text{Gal}(K/Q)$ will be determined by how they act on these roots via permutations, and so will be a subgroup of $S_5$. In fact, the group $S_5$ itself is nonsolvable, since its only normal subgroup is $A_5$, which is itself non-solvable.
		\par If $g$ has a real root $\alpha$, then the minimal polynomial of $\alpha$ over $Q$ will have degree $5$, and $Q(\alpha)$ will be a degree $5$ extension of $Q$. $\alpha$ is also contained in $K$, so by the tower law, $ \left \lvert { K : Q } \right \lvert = \left \lvert { K : Q(\alpha) } \right \lvert \cdot \left \lvert { Q(\alpha) : Q } \right \lvert $. In particular, since $\left \lvert { Q(\alpha) : Q } \right \lvert $ is $5$, the degree of $K$ over $Q$ will be divisible by $5$, and therefore so will the order of its Galois group. By Cauchy, this means there must be some element of degree $5$ in $\text{Gal}(K/Q) \leq S_5$, and the only elements of degree $5$ in $S_5$ are $5$-cycles. So, $\text{Gal}(K/Q)$ must contain a $5$-cycle.
		\par Now, if $g$ is a quintic with $3$ real roots, it must also have $2$ complex roots, which will be complex conjugates. The field automorphism of $C$ which acts by complex conjugation fixes $Q$, and it will descend to an automorphism of $K$ that fixes the three real roots and permutes the pair of conjugates. As an element of $\text{Aut}(K)$ which fixes $Q$, this automorphism is an element of $\text{Gal}(K/Q)$, and since it only acts by swapping two elements, it must have order $2$. Since $\text{Gal}(K/Q)$ is a subgroup of $S_5$, this means there must be a $2$-cycle in it.
		\par As we have seen in homework sets, any $5$-cycle and any $2$-cycle will together generate the symmetric group $S_5$, which we know not to be a solvable group. Therefore, any arbitrary polynomial which is irreducible over $Q$ and has $3$ real roots will have a splitting field whose Galois group is not solvable, and so it will not be solvable by radicals.
		\par One such polynomial is $g(x) = x^5 + 9x^2 - 3$. The prime $3$ divides all coefficients except the leading one, and $3^2$ does not divide the constant term, so by the Eisenstein criterion, $g(x)$ is irreducible. Further, $g$ has $3$ real roots: this can be seen by checking that its derivative is $x^4 + 18x$, which has two zeroes at $0$ and $(-18/5)^{1/3}$, so $g$ has two critical points. At the first, it is positive, and at the second, it is negative, so it changes sign three times: between $-\infty$ and $(-18/5)^{1/3}$, between $(-18/5)^{1/3}$ and $0$, and between $0 $ and $\infty$. Thus it has three real roots and is irreducible over $Q$, which we have seen implies that it is not solvable by radicals.
	\par In order to find a polynomial $f$ which has three real roots and \textit{is} solvable by radicals, $f$ cannot be irreducible over $Q$. For a simple example, we can use $f = (x)(x-1)(x+1)(x^2 + 1)$. This has three real and two complex roots, and it is solvable by radicals (trivially). Or we could take the product of an irreducible polynomial of degree $3$, with $3$ real roots, and an irreducible polynomial of degree $2$, with no real roots: for instance, $f = (x^3 - 4x - 2)(x^2 + 1)$. The first is irreducible by Eisenstein, and the second has no real roots. Since every polynomial of degrees $2$ and  $3$ are solvable by radicals, $f$ must be as well - this follows from the fact that every subgroup of $S_2$ and of $S_3$ is solvable.
	\end{proof}
\end{problem}
\begin{problem}{8}
	let $n$ be an integer greater than $1$, and let $p$ be an odd prime. Perhaps by showing that the roots of $f = X^n + X + p$ are outside the unit disk, prove that $f$ is irreducible over $Q$.
	\begin{proof}
		First, we note that by Gauss' lemma, and the fact that the coefficients of $f$ have no nontrivial common factors, $f$ is irreducible over $Q$ if and only if it is irreducible over $Z$. So, we need only show that $f$ is irreducible over the integers.
		\par We also see that all roots of $f$ lie outside the unit circle: if $\alpha$ is a root of $f$, then it satisfies the equation $\alpha^n + \alpha = -p$. If $\alpha$ lies in the unit circle, then $\alpha^n$ does as well. But by the triangle inequality,
		\[\left \lvert { -p } \right \lvert = \left \lvert { \alpha^n + \alpha } \right \lvert \leq \left \lvert { \alpha^n } \right \lvert + \left \lvert { \alpha } \right \lvert  < 2.\]
		Since $p$ is an odd prime, $\left \lvert { -p } \right \lvert \geq 3$, a contradiction. So, all roots of $f$ lie outside the unit disk in the complex plane.
		\par Now, assume that $f$ is reducible over $Z$, i.e. that it factors into two nonconstant polynomials $g$ and $h$ with integer coefficients. The constant terms of $g$ and $h$ must multiply to the prime $p$, so the constant term of either $g$ or $h$ must be $\pm 1$. But if the constant term of, say, $h$ is $\pm 1$, then the product of the roots of $h$ is $\pm 1$. For this to be true, at least one root of $h$ must lie on or within the unit disk. This root would also be a root of $f$, a contradiction. Therefore, $f$ must be irreducible over the integers, and therefore also over $Q$.
	\end{proof}
\end{problem}
\end{document}
