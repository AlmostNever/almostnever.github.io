% --------------------------------------------------------------
% Andrew Tindall
% --------------------------------------------------------------
 
\documentclass[12pt]{article}
 
\usepackage[margin=1in]{geometry} 
\usepackage{amsmath,amsthm,amssymb,enumitem}
\setlist{
	listparindent=\parindent,
parsep=0pt,}

\newcommand{\N}{\mathbb{N}}
\newcommand{\Q}{\mathbb{Q}}
\newcommand{\Z}{\mathbb{Z}}
\newcommand{\R}{\mathbb{R}}
\newcommand{\mc}[1]{\mathcal{#1}}
\newcommand{\e}{\varepsilon}
\newcommand{\bs}{\backslash}
\newcommand{\PGL}{\text{PGL}}
\newcommand{\Sp}{\text{Sp}}
\newcommand{\tr}{\text{tr}}
\newcommand{\Lie}{\text{Lie}}
\newcommand{\rec}[1]{\frac{1}{#1}}
\newcommand{\toinf}{\rightarrow \infty}


\theoremstyle{definition}
\newtheorem{proofpart}{Part}
\newtheorem{theorem}{Theorem}
\makeatletter
\@addtoreset{proofpart}{theorem}
\makeatother


\newenvironment{problem}[2][Problem]{\begin{trivlist}
\item[\hskip \labelsep {\bfseries #1}\hskip \labelsep {\bfseries #2.}]}{\end{trivlist}}
 
\begin{document}
 
%\renewcommand{\qedsymbol}{\filledbox}
 
\title{Homework 3}
\author{Andrew Tindall\\
	Algebra II}
 
\maketitle
\begin{section}{Problems}
\begin{problem}{1}
Let $R$ be a commutative ring, and let $I$ and $J$ be ideals of $R$. Prove the following:
\begin{enumerate}[label=(\alph*)]
    \item $I + J$ is the smallest ideal of $R$ containing both $I$ and $J$.
    \item $IJ$ is an ideal and it is contained in $I \cap J$.
    \item Give an example withere $IJ \neq I \cap J$.
    \item If $I + J = R$ then $IJ = I \cap J$.
\end{enumerate}
\end{problem}
\begin{proof}
\begin{enumerate}[label=(\alph*)]
    \item It is immediately clear that $I + J \supset I$ and $I + J \supset J$, as any element $i \in I$ can also be written as $i + 0$, where $i \in I$ and $0 \in J$, and equivalently $j = j+0$ for any element $j \in J$.
    \par It also follows quickly that any ideal containing $I$ and $J$ must also contain $I + J$: if $i + j$ is some element of $I + J$, then both $i$ and $j$ must be in $K$, as it contains both ideals. Therefore $i+j$ must be in $K$ as well, as ideals are closed under addition. So any such $K$ must contain $I+J$, and $I + J$ is the minimal element in the set of ideals containing both $I$ and $J$.
    \item We show first that $IJ$ is an ideal. Certainly $0 \in IJ$, since $0 = 0 \cdot 0$. $IJ$ is also closed under addition, as it is defined as the set of all finite sums of monomials $ij$, where $i \in I$ and $j \in J$ - a sum of two such finite sums is also a finite sum. Finally, it is closed under multiplication by any element of $R$: let $a = \sum_K i_kj_k$ be an element of $IJ$, where $k \in K$ is a finite index set.  Then, for any $r \in R$,
    \begin{align*}
        ra &= r \left ( \sum_{k \in K} i_k j_k \right )\\
        &= \sum_{k \in K} r\left (i_k j_k \right )\\
        &= \sum_{k \in K} \left ( r i_k \right ) j_k
    \end{align*}
    Since $I$ is an ideal, $ri_k \in I$, so this is also a finite sum of elements from $I$ and $J$ multiplied together, and is therefore an element of $IJ$. Therefore the set is closed under multiplication by arbitrary elements of $R$, and it is an ideal. (If $R$ were not commutative, and we were checking \textit{left}-ideal axioms, we would need to check that $IJ$ was closed under left-multiplication by arbitrary elements of $R$, and under right-multiplication by other elements of $IJ$, because ideals also need to be subrings. Since $R$ is commutative, the two cases collapse into one, and we only need to check multiplication by generic elements of $R$.)
    \par We now show that $IJ \subset I \cap J$. Let $\sum_K i_k j_k$ be an element of $IJ$. Each term $i_kj_k$ is an element of both $I$ and $J$, and since $I$ and $J$ are closed under finite sums, the element $\sum_K i_k j_k$ must also be in both $I$ and $J$. Therefore any element of $IJ$ is also in $I \cap J$.
    \item We can also see that the inclusion $IJ \subset I \cap J$ need not be an equality. Let $R = k[x,y]$, for $k$ some field, and let $I = \langle x^2 y \rangle$ and $J = \langle x y^2 \rangle$. Then an element of $IJ$ is a sum of monomials, each of which has an exponent greater than or equal to $3$ for both $x$ and $y$ - for example, $k_1 x^7y^5 + k_2 x^5y^3 + k_3 x^3y^3$. However, the element $x^2y^2$ is in both $\langle x^2 y \rangle $ and $\langle xy^2 \rangle$, so it is in $I \cap J$, but it is not in $IJ$.
    \item We now show that if $I+J$ = R, then $IJ = I \cap J$. Since the inclusion has been proved in one direction, we show it in the other: let $a \in I \cap J$ be arbitrary, and we show that it is in $IJ$.
    \par We know that $a \in I$ and $a \in J$. We also assume that $I + J = R$, which means in particular that there exist $i \in I$ and $j \in J$ such that $ i + j = 1$. So, we make the following calculation:
    \begin{align*}
        a &= 1a\\
        &= (i+j)a\\
        &= ia + ja\\
        &= ia +  aj
    \end{align*}
    We see that $ia$ and $aj$ are both in $IJ$, since $a \in J$ and $a \in I$, so the sum $ia + aj \in IJ$ as well. Therefore $IJ \supset I\cap J$, and we can conclude that the two ideals $IJ$ and $I \cap J$ are equal.
\end{enumerate}
\end{proof}
\begin{problem}{2}
Show that if $R$ is an integral domain and $M$ is any nonprincipal ideal of $R$ then $M$ is torsion free of rank $1$ but is not a free $R$-module.
\end{problem}
\begin{proof}
We first show that $M$ is torsion free and its rank is at least $1$, and then that it is at most $1$, and then show that it is not free.
\par It is immediate that it is torsion free, since it is a subring of $R$, and $R$ is an integral domain - if $rm = 0$ for some nonzero $r \in R$ and nonzero $m \in M$, then this relation would also hold in $R$, and it would not be integral.
\par Being torsion free implies that the rank of $M$ is at least $1$: if the rank were $0$, then every element $m$ would be linearly dependent over $R$ - there would be some $r$ such that $rm = 0$, but there is no such $r$, since $M$ is torsion free.
\par Now, let $m_1$ and $m_2$ be any elements of $M$. from $m_2 \cdot m_1 + (-m_1) \cdot m_2 = 0$, we see that there is no linearly independent set of more than $1$ element, so the rank must be exactly $1$.
\par Finally, we show that $M$ is not a free $R$ module. For if it were a free $R$ module, it would necessarily have rank $1$, making it isomorphic (as an $R$-module) to $R$ itself under some isomorphism $\phi$. But then $M$ would be generated as an ideal by the image of $1$ under $\phi$ - to see this, let $m \in M$ be arbitrary; then $m = \phi(r)$ for some $r$, and
\begin{align*}
m &= \phi(r)\\ 
&= \phi(r \cdot 1)\\
&= r \cdot \phi(1)
\end{align*}
Then any $m$ could be written as $r \cdot \phi(1)$ for some $r$, and $M$ would be equal to the principal ideal $\langle \phi(1)\rangle$, contradicting our assumption that it was nonprincipal. So no nonprincipal ideal of an integral domain can be free of rank $1$, despite being torsion free of rank $1$.
\par We show the specific case for $\{2,x\}$ directly in the following problem, but it is also implied by this problem, since $\Z[x]$ is an integral domain and $\{2,x\}$ is a nonprincipal ideal.
\end{proof}
\end{section}
\begin{section}{Extra Stuff}
\begin{problem}{1}
Let $R = \Z[x]$ and let $M = (2,x)$ be the ideal generated by $2$ and $x$, considered as a submodule of $R$. Show that $\{2,x\}$ is not a basis of $M$. Then show that the rank of $M$ is $1$ but that $M$ is not free of rank $1$.
\end{problem}
\begin{proof}
If $\{2,x\}$ were a basis of $M$, then we could write any element $a$ of $M$ as a \textit{unique} sum $f \cdot 2 + g \cdot x$. Of course such a sum always exists, by definition of the generators of an ideal. However, we can show that it is never unique.
\par Let $a = f \cdot 2 + g \cdot x$. Then we can also write $a$ as
\begin{align*}
    a &= f \cdot 2 + g \cdot x + g \cdot x - g \cdot x\\
    &= f \cdot 2 + (g \cdot x) \cdot 2 - g \cdot x\\
    &= (f + g \cdot x) \cdot 2 + (-g) \cdot x
\end{align*}
Therefore $\{2,x\}$ is not a basis, as no element of $\langle 2, x\rangle$ has a unique representation in terms of $2$ and $x$.
\end{proof}
\begin{problem}{2}
If $M$ is a finitely generated module over the P.I.D. $R$, describe the structure of $M / \text{Tor}(M)$.
\end{problem}
\begin{proof}
We show that $M / \text{Tor}(M)$ is a free $R$ module. This follows from the structure theorem for modules over P.I.D.s, as we may decompose $M$ as
\[
M = R^{\oplus n} \oplus \frac{R}{(p_1)} \oplus \dots \oplus \frac{R}{(p_n)}
\]
A theorem in D\&F shows that $\text{Tor}(M) \cong R/(p_1) \oplus \dots \oplus R/(p_n)$, and quotienting the decomposition of $M$ by this submodule gives $M/\text{Tor}(M) \cong R^{\oplus n}$, showing that the torsion-free part of $M$ is a free $R$-module.
\end{proof}
\begin{problem}{3}
Prove that the intersection of all prime ideals of a commutative ring $R$ is equal to its nilradical.
\end{problem}
\begin{proof}
We recall that $\text{nilrad}(R) = \text{rad}(0)$, the set of all elements $f \in R$ such that $f^n = 0$ for some $n \geq 1$. We first show that every such $f$ belongs to every prime ideal $p \in \text{Spec}(R)$, and then that if $f$ is not nilpotent, then there exists some $p \in \text{Spec}(A)$ such that $f \notin p$.
\par It is true in general that, if $f^n \in p$ for some $f \in R$ and $p \in \text{Spec}(R)$, then $f \in p$. In particular, if $f^n = 0$, then $f \in p$. So the nilradical of $R$ is contained in the intersection of all prime ideals of $R$.
\par On the other hand, if $f$ is not a zerodivisor, then the multiplicative set $S = \{1, f, f^2, ... \}$ does not contain $0$. By taking Zorn's lemma\footnote{
If using the Axiom of Choice to prove this simple statement about prime ideals seems exessive, we could also use the Ultrafilter principle,\cite{ultrafilter} which is strictly weaker than Choice, but still stronger than ZF.

}, there exists some maximal ideal $p'$ which does not intersect $S$, and this ideal $p'$ is prime.\cite{reid} Thus there is a prime ideal not containing $f$, and we can conclude that the intersection of all prime ideals is contained in the nilradical of $R$, and thus that the two ideals are equal.
\end{proof}
\end{section}
\begin{thebibliography}{}
\bibitem{ultrafilter}{
PersonX, Nilradicals without Zorn's lemma, URL: https://mathoverflow.net/q/27172
}
\bibitem{reid}{
Reid, Miles. Undergraduate Commutative Algebra. LMS Student Texts 29, London, 1995.
}
\end{thebibliography}

\end{document}
