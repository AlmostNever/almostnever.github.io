% --------------------------------------------------------------
% Andrew Tindall
% --------------------------------------------------------------
 
\documentclass[12pt]{article}
 
\usepackage[margin=1in]{geometry} 
\usepackage{amsmath,amsthm,amssymb,enumitem, tikz-cd}

\newcommand{\N}{\mathbb{N}}
\newcommand{\Q}{\mathbb{Q}}
\newcommand{\Z}{\mathbb{Z}}
\newcommand{\R}{\mathbb{R}}
\newcommand{\mc}[1]{\mathcal{#1}}
\newcommand{\e}{\varepsilon}
\newcommand{\bs}{\backslash}
\newcommand{\PGL}{\text{PGL}}
\newcommand{\Sp}{\text{Sp}}
\newcommand{\tr}{\text{tr}}
\newcommand{\Lie}{\text{Lie}}
\newcommand{\rec}[1]{\frac{1}{#1}}
\newcommand{\toinf}{\rightarrow \infty}


\theoremstyle{definition}
\newtheorem{proofpart}{Part}
\newtheorem{theorem}{Theorem}
\makeatletter
\@addtoreset{proofpart}{theorem}
\makeatother


\newenvironment{problem}[2][Problem]{\begin{trivlist}
\item[\hskip \labelsep {\bfseries #1}\hskip \labelsep {\bfseries #2.}]}{\end{trivlist}}
 
\begin{document}
 
%\renewcommand{\qedsymbol}{\filledbox}
 
\title{Homework 5}
\author{Andrew Tindall}
 
\maketitle
\begin{section}{Problems}
	Assume that $G$ is a finite group.
	\begin{problem}{1}
	Dummit \& Foote, Problem 15.1.3: Prove that the degree $1$ representations of $G$ are in bijective correspondence with the degree $1$ representations of the abelian group $G/G'$ (where $G'$ is the commutator subgroup of $G$)
\end{problem}
\begin{proof}
	A degree $1$ representation of $G$ is a homomorphism $f : G \to \text{GL}_1(F)$, for some field $F$. Because the group $\text{GL}_1(F)$ is isomorphic to the group $F^\times$, it is in particular abelian. By the universal property of the abelianization $G/G'$, there is a unique map $g$ such that the following diagram commutes:
	\[\begin{tikzcd}&G \arrow[r, "f"] \arrow[d, "\pi"] &\ext{GL}_1(F) \\&G/G' \arrow[ur, "g"] \end{tikzcd}\]
	This gives a map from the degree $1$ representations of $G$ to those of $G/G'$. In the other direction, given a map $g : G/G' \to \text{GL}_1(F)$, we may precompose with $\pi: G \to G/G'$ to obtain a map $f: G \to \text{GL}_1(F)$, giving us a map from degree $1$ representations of $G/G'$ to degree $1$ representations of $G$. These maps are inverse, and therefore define a bijection.
\end{proof}
\begin{problem}{2}
	Dummit \& Foote, Problem 15.1.6: write out the matrices $\varphi(g)$ for every $g \in G$ for each of the following representations:
	\begin{enumerate}[label=(\alph*)]
		\item The representation of $S_3$ as permuatations of basis elements in a $3$-dimensional vector space.
		\item The representation of $D_8$ as rotations and reflections of $\R^2$.
		\item The representation of the quaternion group $Q_8$ as transformations of $\mathbb C^2$.
		\item The representation of the quaternion group $Q_8$ as transformations of $\R^4$.
	\end{enumerate}
\end{problem}
\begin{proof}
	\begin{enumerate}[label=(\alph*)]
		\item The permutations in $S_3$ correspond to permutation matrices:
			\begin{align*}
				&e = \begin{bmatrix}
					1 & 0 & 0 \\ 0 & 1 & 0 \\0 & 0 & 1
				\end{bmatrix}  &(12) = \begin{bmatrix}
					0 & 1 & 0 \\1 & 0 & 0\\0 & 0 & 1
				\end{bmatrix}  &(23) = \begin{bmatrix}
					1 & 0 & 0\\0 & 0 & 1\\ 0 & 1 & 0
				\end{bmatrix}\\ &(13) = \begin{bmatrix}
					0 & 0 & 1\\ 0 & 1 & 0 \\ 1 & 0 & 0
				\end{bmatrix} & (123) = \begin{bmatrix}
				0 & 0 & 1 \\1& 0 & 0 \\ 0 & 1 & 0 
			\end{bmatrix} & (132) = \begin{bmatrix}
				0 & 1 & 0\\0 & 0 & 1\\1 & 0 &0 
			\end{bmatrix}
			\end{align*}
		\item The representation of $D_8$ as orthogonal matrices in $\text{GL}_2(\R)$ acts as the symmetries of any square centered at the origin, and with sides either parallel to the coordinate axes or at $45$-degree angles to them. Note that all pure rotations have determinant $1$, while those with a reflection $s$ have determinant $-1$.
			\begin{align*}
				&e = \begin{bmatrix}
					 1 & 0 \\0 & 1
				 \end{bmatrix} &r = \begin{bmatrix}
					 0 & -1 \\ 1 & 0
				 \end{bmatrix} &r^2 = \begin{bmatrix}
					 -1 & 0 \\0 & -1
				 \end{bmatrix} & r^3 = \begin{bmatrix}
					 0 & 1\\-1 & 0
				 \end{bmatrix}\\ & s = \begin{bmatrix}
					 0 & 1 \\1 & 0
				 \end{bmatrix} & sr = \begin{bmatrix}
					 1 & 0 \\ 0 & -1
				 \end{bmatrix} & sr^2 = \begin{bmatrix}
					 0 & -1 \\ -1 & 0
				 \end{bmatrix} & sr^3 = \begin{bmatrix}
					 -1 & 0 \\0 & 1
				 \end{bmatrix}
			\end{align*}
		\item We use the $i,j,k$ presentation of the quaternion group, writing $\bar{i}$ for $i^{-1}$, etc. As $2 \times 2$ complex matrices, the representation of $Q_8$ is:
			\begin{align*}
				&e = \begin{bmatrix}
					1 & 0 \\ 0 & 1
				\end{bmatrix} &i = \begin{bmatrix}
					\sqrt{-1} & 0 \\0 & -\sqrt{1}
				\end{bmatrix} &j = \begin{bmatrix}
					0 & -1 \\ 1 & 0
				\end{bmatrix}
			\end{align*}<++>
	\end{enumerate}<++>
\end{proof}<++>
\end{section}
\end{document}
