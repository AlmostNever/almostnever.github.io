% --------------------------------------------------------------
% Andrew Tindall
% --------------------------------------------------------------
 
\documentclass[12pt]{article}
 
\usepackage[margin=1in]{geometry} 
\usepackage{amsmath,amsthm,amssymb,enumitem}

\newcommand{\N}{\mathbb{N}}
\newcommand{\Q}{\mathbb{Q}}
\newcommand{\Z}{\mathbb{Z}}
\newcommand{\R}{\mathbb{R}}
\newcommand{\mc}[1]{\mathcal{#1}}
\newcommand{\e}{\varepsilon}
\newcommand{\bs}{\backslash}
\newcommand{\PGL}{\text{PGL}}
\newcommand{\Sp}{\text{Sp}}
\newcommand{\tr}{\text{tr}}
\newcommand{\Lie}{\text{Lie}}
\newcommand{\rec}[1]{\frac{1}{#1}}
\newcommand{\toinf}{\rightarrow \infty}


\theoremstyle{definition}
\newtheorem{proofpart}{Part}
\newtheorem{theorem}{Theorem}
\makeatletter
\@addtoreset{proofpart}{theorem}
\makeatother


\newenvironment{problem}[2][Problem]{\begin{trivlist}
\item[\hskip \labelsep {\bfseries #1}\hskip \labelsep {\bfseries #2.}]}{\end{trivlist}}
 
\begin{document}
 
%\renewcommand{\qedsymbol}{\filledbox}
 
\title{Homework 11}
\author{Andrew Tindall\\
Algebra II}
 
\maketitle
\begin{problem}{1}
	Dummit \& Foote, 10.5.5: (for an arbitray finite index set): Let \[\mathfrak m = (f_1(x_1), f_2(x_1,x_2), \dots , f_n(x_1, \dots x_n) )\] be a maximal ideal in $k[x_1, \dots x_n]$, where $f_1, \dots f_n$ are irreducible polynomials such that $f_i$ is irreducible modulo $f_1, \dots f_{i-1}$. Show that $K = k[x_1, \dots x_n]/\mathfrak m$ is an algebraic field extension of $k$, so that $k[x_1, \dots x_n]$ can also be viewed as a subring of $K[x_1, \dots x_n]$. If $x_1, \dots x_n$ are mapped to $\alpha_1, \dots, \alpha_n$ respectively, under the canonical homomorphism $k[x_1, \dots x_n] \to k[x_1, \dots x_n]/\mathfrak m$, prove that $\mathfrak m = k[x_1, \dots x_n] \cap (x_1 - \alpha_1, \dots , x_n - \alpha_n) \subset K[x_1, \dots x_n]$.
	\par This is something of an extension of the weak Nullstellensatz to a not-necessarily algebraically closed field.
	\begin{proof}
		We first show that $K = k[x_1, \dots x_n]/\mathfrak m$ is an algebraic field extension of $k$. It suffices to show that it is a finite field extension, which it is. 
		\par We proceed by induction. First, we know that for $n = 1$, the ring $K_1 = k[x]/(f_1(x))$ is a finite field extension of $k$: it is a field, because $(f_1(x))$ is maximal, it contains $k$ as a subfield, and it is generated as a $k$-vector space by the elements $x, x^2, \dots x^{\text{deg}(f) - 1}$.
		\par Now, assume that we know the field $K_{i-1} = k[x_1, \dots , x_{i-1}]/(f_1(x_1), \dots f_{i-1}(x_1, \dots x_{i-1}))$ is a finite field extension of $k$. We wish to show that $K_i = k[x_1, \dots x_i]/(f_1, \dots f_i)$ is a finite field extension of $k$. Because $(f_1, \dots f_i)$ is an ideal of $k[x_1, \dots x_i]$ containing $(f_1, \dots f_{i-1})$, the third isomorphism theorem for rings gives us
		\begin{align*}
			\frac{k[x_1, \dots x_i]}{(f_1, \dots f_i)} \cong \frac{(k[x_1, \dots x_i])/ (f_1, \dots f_{i-1})}{(f_1, \dots f_i)/ (f_1, \dots f_{i-1})}\\
		\end{align*}
		Since $x_i$ does not appear in the polynomials $f_1, \dots f_{i-1}$, we have the isomorphism \[k[x_1, \dots x_i]/ (f_1, \dots f_{i-1}) \cong (k[x_1, \dots x_{i-1}]/(f_1, \dots f_{i-1}))[x_i] = K_{i-1}[x_i].\] 
		\par Let $\overline{f_i} $ be the image of $f_i$ modulo $f_1, \dots f_{i-1}$. Using the above isomorphim, we have 
		\[K_i \cong K_{i-1}[x_i]/(\overline{f_i}(x_1, \dots x_i)),\]
		\par Where $\overline{f_i}(x_1, \dots x_i)$ is considered as an irreducible polynomial in $K_{i-1}[x_i]$. Using the same argument as in the $1$-variable case, $K_i$ is a finite field extension of $K_{i-1}$, and is therefore a finite field extension of $k$, of degree $[K_i : k] = [K_i : K_{i-1}] \cdot [K_{i-1} : k]$.
		\par Using the field extension $k \hookrightarrow K$, we can view $k[x_1, \dots x_n]$ as a subring of $K[x_1, \dots x_n]$. 
		\par \; 
		\par The second half of this problem is unfinished.
		\par \;
	\end{proof}
\end{problem}
\begin{problem}{2}
	Dummit \& Foote, 10.5.7: Let $(f) = (x^5 + x + 1)$ in $\text{Spec}\Z[x]$ viewed as fibered over $\text{Spec}\Z$ as in Example $3$ following Proposition $55$. Prove that there are two closed points in the fiber over $(2)$, three closed points in the fiber over $5$, four closed points in the fiber over $(19)$, and five closed points in the fiber over $(211)$.
	\begin{proof}
		The proper method for this solution looks like it involves an interesting application of Galois theory (following chapter 14.8 in Dummit \& Foote), but it was a lot simpler to factor these over finite fields using a CAS (Wolfram Alpha).
		\begin{itemize}
			\item $(2)$: Modulo $2$, we have
				\begin{align*}
					(x^2 + x + 1)(x^3 + x^2 + 1) &= x^5 + 2x^4 + 2x^3 + 2x^2 + x + 1\\
					&\equiv x^5 + x + 1
				\end{align*}
				Where $x^2 + x + 1$ and $x^3 + x^2 + 1$ are both irreducible modulo $2$, so there are exactly two points in the fiber over $(2)$, corresponding to the ideals $(2, x^2 + x + 1)$ and $(2, x^3 + x^2 + 1)$.
				\par Proof that $x^2 + x + 1$ is irreducible modulo $2$:
				\par Because there are only $2$ linear polynomials modulo $2$, if $x^2 + x + 1$ were reducible, it would have to have either $x$ or $x+1$ as a factor. Because it has a nonzero constant term, it cannot have $x$ as a factor, so it would have to be equal to $(x+1)^2$. However, $(x+1)^2 \equiv x^2 + 1 \not\equiv x^2 + x +1$. So, this polynomial is not reducible.
				\par Proof that $x^3 + x^2 + 1$ is irreducible modulo $2$:
				\par If $x^3 + x^2 + 1$ were reducible and had only linear factors, it would again have no factors of $x$, because it has nonzero constant term. Therefore, it would need to be equal to $(x+1)^3$. However, $(x+1)^3 \equiv x^3 + x^2 + x + 1 \not\equiv x^3 + x^2 + 1$.
				\par This leaves the possibility that $x^3 + x^2 + 1$ is reducible, and has a linear factor and a quadratic irreducible factor. There is only one irreducible quadratic modulo $2$, and there is only one possible linear factor, so the only possibility is $(x^2 + x + 1)(x+1)$. However, $(x^2 + x + 1)(x+1) \equiv x^3 + 1 \not\equiv x^3 + x^2 + 1$. So, this is an irreducible cubic modulo $2$.
			\item $(5)$: Modulo $5$, we have 
				\begin{align*}
					(x + 3) (x^2 + x + 1) (x^2 + x + 2) &= (x^3 + 4x^2 + 4x + 3)(x^2 + x + 2)\\
					&= x^5 + 5x^4 + 5x^3 + 11x + 6\\
					&\equiv x^5 + x + 1
				\end{align*}
				Where $x^2 + x + 1$ and $x^2 + x + 2$ are irreducible modulo $5$. So, there are $3$ points in the fiber over $5$, corresponding to $(5, x+3)$, $(5, x^2 + x + 1)$, and $(5, x^2 + x + 2)$.
				\par Proof that $x^2 + x + 1$ is irreducible modulo $5$: if $x^2 + x + 1$ had linear factors modulo $5$, their constant terms would need to multiply to give $1$. The possible pairs are:
				\begin{itemize}
					\item $1, 1$: we have $(x+1)^2 \equiv x^2 + 2x + 1 \not\equiv x^2 + x + 1$.
					\item $2, 3$: we have $(x+2)(x+3) \equiv x^2 + 1 \not\equiv x^2 + x + 1$.
					\item $4, 4$: we have $(x+4)(x+4) \equiv x^2 + 3x + 1 \not\equiv x^2 + x + 1$.
				\end{itemize}
				So, $x^2 + x + 1$ is irreducible modulo $5$.
				\par Proof that $x^2 + x + 2$ is irreducible modulo $5$: If this polynomial had $2$ linear factors modulo $5$, their constant terms would need to multiply to give $2$. The possible pairs are:
				\begin{itemize}
					\item $1, 2$: we have $(x+1)(x+2)\equiv x^2 + 3x + 2 \not\equiv x^2 + x + 2$.
					\item $3, 4$: we have $(x+3)(x+4) \equiv x^2 + 2x + 2 \not\equiv x^2 +x +2$. 
				\end{itemize}
				So, $x^2 + x + 2$ is an irreducible quadratic modulo $5$.
			\item $(211)$: Modulo $211$, we have
				\begin{align*}
					(x + 15) (x + 35) (x + 51) (x + 124) (x + 197) &= x^5 + 422 x^4 + 59924 x^3 \\&\;+ 3481078 x^2 + 83710875 x + 654059700 \\
					&\;\\
					&= x^5 + (2 * 211)x^4 + (284 * 211)x^3 + \\&\;(16598 * 211)x^2 + (396734 * 211 + 1)x \\&\;+ 3099809 * 211 + 1 \\&\;\\
					&\equiv x^5 + x + 1
				\end{align*}
				So, there are $5$ points in the fiber over $(211)$, corresponding to the $5$ linear factors of the polynomial $x^5 + x + 1$ modulo $211$.
		\end{itemize}
	\end{proof}
\end{problem}
\end{document}
