% --------------------------------------------------------------
% Andrew Tindall
% --------------------------------------------------------------
 
\documentclass[12pt]{article}
 
\usepackage[margin=1in]{geometry} 
\usepackage{amsmath,amsthm,amssymb,enumitem}

\newcommand{\N}{\mathbb{N}}
\newcommand{\Q}{\mathbb{Q}}
\newcommand{\Z}{\mathbb{Z}}
\newcommand{\R}{\mathbb{R}}
\newcommand{\mc}[1]{\mathcal{#1}}
\newcommand{\e}{\varepsilon}
\newcommand{\bs}{\backslash}
\newcommand{\PGL}{\text{PGL}}
\newcommand{\Sp}{\text{Sp}}
\newcommand{\tr}{\text{tr}}
\newcommand{\Lie}{\text{Lie}}
\newcommand{\rec}[1]{\frac{1}{#1}}
\newcommand{\toinf}{\rightarrow \infty}


\theoremstyle{definition}
\newtheorem{proofpart}{Part}
\newtheorem{theorem}{Theorem}
\makeatletter
\@addtoreset{proofpart}{theorem}
\makeatother


\newenvironment{problem}[2][Problem]{\begin{trivlist}
\item[\hskip \labelsep {\bfseries #1}\hskip \labelsep {\bfseries #2.}]}{\end{trivlist}}
 
\begin{document}
 
%\renewcommand{\qedsymbol}{\filledbox}
 
\title{Homework 8}
\author{Andrew Tindall\\
Analysis I}
 
\maketitle
\begin{problem}{1}
	Let $F : A \to \R$ be a continuous function on a compact set $A \subset \R^n$. Verify that, setting:
	\[F(x) := \min_{y \in A}\left\{ F(y) + \frac{\left \lvert { x - y } \right \lvert }{\text{dist}(x, A)} - 1 \right\} \quad \text{for all }x \in \R^n \backslash A,\]
	defines a continuous extension of $F$ on $\R^n$.
	\begin{proof}
		\textit{incomplete.}
	\end{proof}
\end{problem}
\begin{problem}{2}
	Let $(X, \mathcal{M}, \mu)$ be a measure space. For every $A \subset X$, define:
	\[\mu^*(A) := \inf\left\{ \mu(B); A \subset B, B \in M \right\}\]
	\begin{enumerate}[label=(\roman*)]
		\item Show that $\mu^*$ is a measure generator, coinciding with $\mu$ on $\mathcal{M}$ and such that it is $0$ on every subset of a zero $\mu$-measure set.
		\item Let $\mathcal{M}_c$ be the $\sigma$-algebra generated by $\mu^*$. Show that $\mathcal{M} \subset \mathcal{M}_c$.
		\item Is the following characterization true?:
			\[\mathcal{M}_c = \left\{ A \in 2^X; \exists B \in \mathcal{M} \; A \subset B \text{ and } \mu(B) = \mu^*(A) \text{ and } \mu^*(B \bs A) = 0 \right\}\]
	\end{enumerate}
			\begin{proof}
				\begin{enumerate}[label = (\roman*)]
					\item \textit{incomplete}
					\item \textit{incomplete}
					\item \textit{incomplete}
				\end{enumerate}
			\end{proof}
\end{problem}
\begin{problem}{3}
	Let $f: [a, b] \to \R$ be a given function. 
	\begin{enumerate}[label=(\roman*)]
		\item If $f$ is continuous, show that its graph is a set of (Lesbesgue) measure $0$ in $\R^2$.
		\item What if $f$ is just a (possibly discontinuous) monotone function?
	\end{enumerate}
	\begin{proof}
		\begin{enumerate}[label=(\roman*)]
			\item Let $\varepsilon > 0$. We show that the graph of $f$ may be covered with a measurable subset of $\R^2$ with measure $\leq \varepsilon$.
				Because $f$ is a continuous function on the compact set $[a,b]$, it is uniformly continuous. Therefore, there must be some $\delta$ such that, for any $x \in [a,b]$, for all $y \in [a,b]$ such that $\left \lvert { x-y } \right \lvert < \delta$, $\left \lvert { f(x) - f(y) } \right \lvert < \frac{\varepsilon}{2(b-a)}$. This implies that, for any $x \in [a,b]$, the graph of $f$ restricted to $[x, x+\delta]$ can be covered by the rectangle $[x, x + \delta] \times [f(x) - \frac{\varepsilon}{2(b-a)}, f(x) + \frac{\varepsilon}{2(b-a)}]$. Thus, dividing $[a,b]$ into the intervals
				\[ [a, a + \delta] \cup [a + \delta, a + 2\delta] \cup \dots \cup [a + n\delta, b],\]
				we can cover the graph of $f$ with the rectangles
				\begin{align*}[a, a + \delta] \times \left[f(a) - \frac{\varepsilon}{2(b-a)}, f(a) + \frac{\varepsilon}{2(b-a)}\right]\cup\\ [a + \delta, a + 2\delta] \times \left [f(a + \delta) - \frac{\varepsilon}{2(b-a)}, f(a+\delta) + \frac{\varepsilon}{2(b-a)} \right ]\cup \dots\end{align*}
				Each of these rectangles has a height of $\frac{\varepsilon}{(b-a)}$, and their widths add up to $b-a$, so the total area covered is $\frac{\varepsilon}{b-a} \cdot (b-a)$, or $\varepsilon$. Since $\varepsilon$ was arbitrary, the graph of the function must have measure $0$.
			\item The graph of any monotone function also has measure $0$. Assume $f$ is some monotone increasing function on $[a,b]$, and let $h = f(b) - f(a)$. If $f$ is constant, its graph clearly has measure $0$, so assume that $h > 0$. Let $\varepsilon > 0$ be arbitrary.
				As $f$ is a monotone function on a compact set, it can only have countably many points of discontinuity. If the set of points of discontinuity is empty, then $f$ is continuous, and its graph has measure $0$ by the last problem. Assuming that the set is nonempty and infinite, let $\left\{ x_n\right\}_{n=1}^\infty$ be an enumeration of the points of discontinuity. Around point $x_n$, we put the box $\left[ x_n - \frac{\varepsilon}{2^{n+2}h}, x_1 + \frac{\varepsilon}{2^{n+2}h} \right]\times [f(a), f(b)]$. Each box has height $h$, and the sum of their widths is
				\begin{align*}
					\sum_{n=1}^\infty \frac{\varepsilon}{2^{n+1}h} &= \frac{\varepsilon}{2}
				\end{align*}
				Further, $f$ is continuous on the remaining parts of $[a,b]$, and so its graph can be covered by a union of boxes with measure $\varepsilon/2$ as well (this should be proven, but it is true). Thus the whole graph of $f$ can be covered with a set of measure $\varepsilon$, where $\varepsilon$ is arbitrary, meaning that its graph must have measure $0$.
				\par 				If the points of discontinuity are finite in number, say there are $m$ of them, the proof can be done similarly; the width of each box can be chosen to be $\frac{\varepsilon}{2mh}$, again covering the discontinuities with a set of measure $\frac{\varepsilon}{2}$. In this case the remaining part of $[a,b]$ can be covered with finitely many intervals on which $f$ is continuous, and so its graph on these intervals has measure $0$ by the last problem.
		\end{enumerate}
	\end{proof}
\end{problem}
\begin{problem}{4}
	Prove that the following subsets of $[0,1]$ are compact, of Lesbesgue measure $0$, and uncountable:
	\begin{enumerate}[label=(\roman*)]
		\item The set $A$ containing all numbers which admit a binary representation $0.c_1c_2c_3\dots$ such that $c_n=0$ for all $n$ odd,
		\item The set $B$ of all numbers which admit a binary representation $0.c_1c_2c_3\dots$ such that for every $n$ there is: $c_n = 0$ or $c_{n+1} = 0$.
	\end{enumerate}
	\begin{proof}
		\begin{enumerate}[label=(\roman*)]
			\item First, we show that the set is compact. Because it is a subset of a compact set, it is enough to show that it is closed. Let $\{x^n\}$ be a Cauchy sequence of elements of $A$. We show that the limit $x$ of the sequence is also an element of $A$.
			\par We note first that the set $A$ can equivalently be defined as the set of numbers in $[0,1]$ whose base-$4$ expansion can be written with only $0$s and $2$. Because a sequence of such numbers contains no $3$s in any of their quaternary expansions, the limit $x$ must have a unique expression in base-$4$ (there can be no "trailing $3$s"), and, as with decimal numbers, a sequence of such numbers must "settle" at each digit: the $m$th quaternary digit of each $x^n$, for all $n$ greater than some $N$, must be constant, and equal to the $m$th digit of the limit $x$. Therefore, the quaternary decimal expansion of $x$ must be composed of only $0$s and $2$s, so it must also be an element of $A$.
			\par Now, we show that the measure of $A$ is $0$. Because $A$ is compact, it is measurable, so it suffices to show that the complement of $A$ in $[0,1]$ has measure $1$. 
			\par The complement of $A$ consists of all those numbers which have at least one $1$ or one $3$ in their quaternary expansion. This can be broken up into a countable union: let $A^c_1$ be the set
			\[ [0.1_4, 0.2_4) \cup (0.3_4, 1],\]
			Let $A^c_2$ be the set
			\[ [0.01_4, 0.02_4) \cup (0.03_4, 0.01_4] \cup [0.21_4, 0.22_4) \cup (0.23_4, 0.3_4],\]
			And so on. Each set $A^c_n$ is dis joint from any other $A^c_m$, and each $A^c_n$ has measure $2^n \cdot \frac{1}{2^{n+1}}$. Finally, every element of $A^c_n$ has at least one $1$ or one $3$ in its quaternary expansion, meaning that $\bigcup_n A^c_n \subset A^c$ (in fact, they are equal). Therefore the measure of $A^c$ is greater than or equal to the sum of the measures of the $A^c_n$, which is 
			\[\sum_{n=1}^\infty \frac{1}{2^n} = 1\]
			Because the complement of $A^c$ is $A$, which is measurable, $A$ must have measure $1-1 = 0$.
		\item The set $B$ can also be described in base $4$, a little less succintly: $B$ consists of all those numbers which have only $0$, $1$, and $2$ in their base $4$ expansion, and where a $1$ is never followed by a $2$ (because $12_4$ in base 2 is $110$, and there cannot be any $11$s in the base $2$ expansions). 
			\par $B$ is compact by the same argument as for $A$: because there are no $3$s in the base $4$ expansion of any of its elements, any Cauchy sequence of elements of $B$ will have eventually constant digits in base $4$, meaning that the digits of its limit follow the same rules, and its limit is in $B$. 
			\par It is uncountable because it contains an uncountable subset: let $B'$ be the set of numbers in $[0,1]$ whose base $4$ expansions have only $0$ and $1$, and do not have trailing $1$s. This can be put in an obvious correspondence with the set of numbers in $[0,1]$ whose base $2$ expansions have only $0$ and $1$ and do not have trailing $1$s, which is in fact all numbers in $[0,1]$ - an uncountable set. Because $B' \subset B$, we see that $B$ is also uncountable.
			\par Measure $0$: \textit{incomplete}
		\end{enumerate}
	\end{proof}

\end{problem}
\begin{problem}{5}
	Show that the derivative of a differentiable function $f: (a, b)\to \R$ is a (Lesbesgue) measurable function.
	\begin{proof}
		This follows quickly from the fact that a pointwise limit of measurable functions is measurable (\cite{rudin}), and the fact that the sequence
		\[f_n(x) = \frac{f(x + 1/n) - f(x)}{1/n}\]
		Converges pointwise to the derivative of $f$, for any differentiable function. Because $f_n$ is formed by adding measurable functions and multiplying by scalars, each $f_n$ is measurable, and therefore their pontwise limit, which is the derivative of $f$, is also measurable.
		\par Source: \cite{se}
	\end{proof}
\end{problem}
\begin{thebibliography}{}
	\bibitem{se}{Henning Makholm, Is the derivative of differentiable function $f:\mathbb{R}\to\mathbb{R}$ measurable on $\mathbb{R}$?, URL: https://math.stackexchange.com/q/1803668}
	\bibitem{rudin}{Rudin, W, Real and Complex Analysis, Third Edition. McGraw Hill, 1987}
\end{thebibliography}<++>
\end{document}
