% --------------------------------------------------------------
% Andrew Tindall
% --------------------------------------------------------------
 
\documentclass[12pt]{article}
 
\usepackage[margin=1in]{geometry} 
\usepackage{amsmath,amsthm,amssymb,enumitem,hyperref,tikz-cd}

\newcommand{\N}{\mathbb{N}}
\newcommand{\Q}{\mathbb{Q}}
\newcommand{\Z}{\mathbb{Z}}
\newcommand{\C}{\mathbb{C}}
\newcommand{\R}{\mathbb{R}}
\renewcommand{\P}{\mathbb{P}}
\newcommand{\mc}[1]{\mathcal{#1}}
\newcommand{\e}{\varepsilon}
\newcommand{\bs}{\backslash}
\newcommand{\PGL}{\text{PGL}}
\newcommand{\Sp}{\text{Sp}}
\newcommand{\tr}{\text{tr}}
\newcommand{\Lie}{\text{Lie}}
\newcommand{\rec}[1]{\frac{1}{#1}}
\newcommand{\toinf}{\rightarrow \infty}


\theoremstyle{definition}
\newtheorem{proofpart}{Part}
\newtheorem{theorem}{Theorem}
\makeatletter
\@addtoreset{proofpart}{theorem}
\makeatother


\newenvironment{problem}[2][Problem]{\begin{trivlist}
\item[\hskip \labelsep {\bfseries #1}\hskip \labelsep {\bfseries #2.}]}{\end{trivlist}}
 
\begin{document}
 
%\renewcommand{\qedsymbol}{\filledbox}
 
\title{Final Exam}
\author{Andrew Tindall\\Algebraic Geometry}
\begin{problem}{1}
State the following:
\begin{enumerate}[label=(\alph*)]
	\item Definitions of a ringed space and abstract variety.
		\begin{proof}	
			A \textit{ringed space} is a pair $(X, \mathcal{O}_X)$ of a topological space $X$, along with a sheaf of rings $\mathcal{O}_X$ on $X$. More precisely, a sheaf of rings on a space is a correspondence $U \mapsto \mathcal{O}_X$ of a ring - usually assumed commutative - for each open set $U$ of $X$, along with, for each inclusion $V \subset U$ of open sets, a map $\rho_{U,V}: \mathcal{O}_U \to \mathcal{O}_V$. The maps must satisfy the following property: for any chain $W \subset V \subset U$ of open sets of $X$,
			\[ \rho_{W,V} \circ \varphi_{V,U} \equiv \rho_{W,U}.\]
			We usually write $\rho_{U,V}(f) = f\lvert_V$. The above requirements make $\mathcal{O}_X$ a presheaf. Sheafness is accomplished by the final two requirements:
			\par Take any open covering $\bigcup_i U_i = U$ of an open set. If we have a family of elements $f_i \in \mathcal O_{U_i}$ such that $f_i\lvert_{U_i \cap U_j} = f_j\lvert_{U_i \cap U_j}$, then there exists a unique $f \in \mathcal O_U$ such that $f\lvert_{U_i} = f_i$ for all $i$.
			\par A morphism of ringed spaces is a continuous map $ \varphi: X \to Y$ and, for each open set $V$ of $Y$, a map $\varphi(V)\mathcal{O}_{Y,V} \to \mathcal{O}_{X,\varphi^{-1}(V)}$, such that, for any $U \subset V$ of $Y$, the following square commutes:
			\[\begin{tikzcd}&\mathcal{O}_{Y,V} \arrow[r, "\varphi^*(V)"] \arrow[d, "\rho_{V,U}"] & \mathcal{O}_{X, \varphi^{-1}(V)} \arrow[d, " \rho_{\varphi^{-1}(V), \varphi^{-1}(U)}"]\\& \mathcal{O}_{Y,U} \arrow[r, "\varphi^*(U)"] & \mathcal{O}_{X, \varphi^{-1}(U)}\end{tikzcd}\]
			An isomorphism of ringed spaces is a morphism with a two-sided inverse.
			\par We also need the definition of the spectrum of a ring $R$, as a ringed space. The topological space $X = \text{Spec}(R)$ is the set of prime ideals of $R$, with the Zariski topology, which is generated by the distinguished opens $D(f)$; the sheaf of rings $\mathcal{O}_X$ is defined on distinguished opens by $\mathcal O_{X, D(f)} = R_f$, the localization of $R$ at $f$. This behaves well with respect to colimits, so it extends to a definition a generic open set $U$ of $X$ by covering it with distinguished opens $D(f_i)$ and taking the colimit $\lim_{\to} D(f_i)$ of the cover. The upshot is that this is a ringed space, where the original ring can be recovered by $\Gamma(X, \mathcal{O}_X) = R$.
			\par Now, we define an abstract variety. First, the definition of an abstract \textit{prevariety}: a ringed space $(X, \mathcal{O}_X)$ is an abstract prevariety if it is quasi-compact - every open cover has a finite subcover - and if, for each $x \in X$, there is some open set $U$ of $X$ containing $x$, such that $(U, \mathcal{O}_{X, U})$ is isomorphic (as a ringed space) to the spectrum of some ring $R$.
				\par An abstract variety is an abstract prevariety which satisfies the following separation axiom: For If $Z$ is an affine variety, and $\varphi_1, \varphi_2$ are morphisms $Z \to X$, then the set $\left\{ z \in Z \; \mid \; \varphi_1(z) = \varphi_2(z) \right\}$ (the preimage $(\varphi_1 \times \varphi_2)(\Delta)$ of the diagonal of $X \times X$) is a closed subset of $Z$.
		\end{proof}
	\item Hilbert's theorem on Hilbert polynomial and degree, and the BKK theorem on the number of solutions of a system of Laurent polynomial equations.
		\begin{proof}
			First, we state Hilbert's theorem. Let $X \subset \mathbb P^N$ be a projective variety with corresponding ideal $I$, and homogenous coordinate ring $k[X] = k[x_0, \dots , x_N]/I$. Because $I$ is homogeneous, it is graded, and so this is a graded ring:
			\[k[X] = \bigoplus_{m \geq 0} k[X]_m\]
			Where each component $k[X]_m$ is defined to be $k[x_0, \dots , x_N]_m $ mod $I$: the subspace generated by the monomials in the $x_i$ which have degree $m$, modulo the ideal $I$. The quantity $H_X(m)$ is defined as the dimension of the component $k[X]_m$ as a $k$-vector space: 
			\[H_X(m) = \text{dim}_k(k[X]_m)\]
			This is known as the Hilbert polynomial, because it is in fact polynomial in $m$: the first conclusion of Hilbert's theorem is that there is a unique polynomial $P_X(t)$ such that $P_X(m) = \text{dim}_k k[V]_m$ for all $m$ sufficiently large.
			\par Next, Hilbert's Theorem states that the degree of $P_X$ is equal to the dimension of $X$: 
			\[\text{deg}(P_X(m)) = \text{dim}(X)\]
			\par Finally, Hilbert's theorem states that, if $a_r$ is the leading coefficient of $P_X(m)$, then $\text{deg}(X) = r!a_r$, where the degree $\text{deg}(X)$ of the variety $X$ is equal to $\left \lvert { X \cap L } \right \lvert $, for $L$ a generic plane in $\mathbb P^N$ whose dimension is equal to the dimension of $X$.
			\par  Next, we state the BKK theorem. Let $A$ be a finite set of elements of $\Z^n$, so that 
			\[A = \left\{ \alpha_0, \dots , \alpha_N \right\} \subset \Z^n\]
			We recall that the definition of $x^\alpha$, for $x = \langle x_1, \dots , x_n\rangle$ and $ \alpha = \langle \alpha_1, \dots , \alpha_n\rangle$, is that
			\[x^\alpha := x_1^{\alpha_1} \cdot \cdots \cdot x_n^{\alpha_n}.\]
			\par Now, let $\mathcal L_A$ be the span $\left\{ x^{\alpha_0}, \dots , x^{\alpha_N} \right\}$, in the $k$-algebra of Laurent polynomials in the variables $x_1, \dots , x_n$. Since each $\alpha_i$ is distinct, the dimension of $\mathcal{L}_A$ as a $k$-algebra is $\left \lvert { A } \right \lvert $, which is $N + 1$.
			\par Let $(f_1, \dots , f_n)$ be a generic element of $\mathcal{L}_A \times \cdots \times \mathcal{L}_A$, the product of $n$ copies of $\mathcal{L}_A$. then the number of points of $(k \backslash 0)^n$ which satisfy each polynomial $f_1, \dots , f_n$ simultaneously is \textit{independent} of the (generic) choice of $f_1, \dots , f_n$, and is equal to $n! \text{vol}_n(\text{conv}(A))$, where $\text{vol}_n(\text{conv}(A))$ is Euclidean $n$-volume of the convex hull $\text{conv}(A) \subset \R^n$ containing the points in $A$.
		\end{proof}
\end{enumerate}
\end{problem}
\begin{problem}{2}
	Let $q_0, \dots, q_N$ be positive integers with $\text{gcd}(q_0, \dots , q_N) = 1$. Define the \textit{weighted projective space} $\mathbb P(q_0, \dots, q_N)$ (as a set) by:
	\[\mathbb P(q_0, \dots , q_N) = (\mathbb C^{N+1} \backslash \left\{ 0 \right\}) / \sim,\]
	where $\sim$ is the equivalence relation given by: 
	\[(x_0, \dots, x_N) \sim (y_0, \dots , y_N) \iff \exists 0 \neq \lambda \in \mathbf k, x_i = \lambda^{q_i}y_i, \forall i = 0, \dots , N.\]
	Similarly to the projective space, we denote the equivalence class of $(x_0, \dots , x_N)$ by $(x_0 : \cdots : x_N)$.
\par Consider the weighted projective space $\mathbb P(1,1,2)$ and the map $\Phi: \mathbb P(1,1,2) \to \mathbb P^3$ given by:
\[(x_0 : x_1 : x_2 ) \mapsto (x_0^2:x_0x_1 : x_1^2 : x_2).\]
\begin{enumerate}[label=(\alph*)]
	\item Show that this map is one-to-one and its image $X = \text{Im}(\Phi)$ is a closed subvariety. Find the defining homogeneous equation of $X \subset \mathbb P^3$.
		\begin{proof}
			We note that the usual ``rescaling'' that one can do in $\P^n$ can be done here as well, with some slight changes. If a coordinate $x_i$ is nonzero, then 
			\[(x_0:\cdots:x_i: \cdots : x_n) \sim ( (x_i)^{-\alpha_0/\alpha_i}x_0 : \cdots : 1 : \cdots : (x_i)^{-\alpha_0/\alpha_i}).\]
			Since we are working in the algebraically closed field $\C$, we do not have trouble with fractional exponents, except that there is not necessarily a unique choice for $x_i^{1/\alpha_i}$. This is not any trouble in the following proofs, although we have to be careful not to assume that the coordinates of $x$ are unique after renormalizing.
			To begin with, we show that the map is one-to-one. Let $(x_0:x_1:x_2)$ and $(x_0':x_1':x_2')$ be two points which map to the same point $(y_0, y_1, y_2, y_3)$ under $\Phi$. First, assume that $y_3 \neq 0$, and rescale so that $y_3 = x_2 = x_2' = 1$.
			\par We know that $x_0^2 = x_0'^2$, so $x_0 = \pm x_0'$, and similarly $x_1 = \pm x_1'$. Also, if $x_0 = -x_0$, then $x_0x_1 = x_0'x_1'$ implies that $x_1 = -x_1'$ as well. So either $(x_0, x_1) =  (x_0', x_1')$ or $(-x_0', -x_1')$. 
			\par But in fact these two cases both correspond to the same point in $\mathbb P(1,1,2)$:
			\begin{align*}
				(-x_0: -x_1 : 1) &= (-1 (-x_0) : -1(-x_0) : (-1)^2)\\
				&= (x_0:x_1:1)
			\end{align*}
			So, the two points are equal.
			\par If $y_3 = 0$, then both $x_2$ and $x_2'$ are zero as well. At least one of $y_0, y_2$ must be nonzero, since if they were both zero, then so would $y_1$. So, say $y_2$ \neq 0. Rescale so that $x_1^2 = x_1'^2 = 1$. 
			\par Again, by $x_0^2 = x_0'^2$ and $x_1^2 = x_1'^2$, along with $x_0x_1 = x_0'x_1'$, we can deduce that $(x_0, x_1) = (\pm x_0', \pm x_1')$,and in both cases we have $(x_0: x_1 : 0) = (x_0':x_1':0)$. By a symmetric argument for $y_1 \neq 0$, we have exhausted all cases. So indeed, any two points in $\P(1,1,2)$ that map to the same point under $\Phi$ must be equal, and so the map is one-to-one.
			\par Next, we want to see that $\text{Im}(\Phi)$ is a closed subvariety. We will see that the homogeneous polynomial
		\[f(y_0, y_1, y_2, y_3) = y_0y_2 - y_1^2\]
		is the defining equation of $\text{Im}(\Phi)$. First, we see that the equation is $0$ for any point $\Phi(x_0: x_1: x_2)$:
		\begin{align*}
			f(\Phi(x_0:x_1:x_2)) &= f(x_0^2: x_0x_1: x_1^2:x_2)\\
			&= x_0^2 x_1^2 - (x_0x_1)^2\\
			&= 0
		\end{align*}
		Now, let $y = (y_0:y_1:y_2:y_3) \in \P^3$ be a point such that $f(y) = 0$. Pick some square root $x_0$ of $y_0$, so that $x_0^2 = y_0$. If $x_0 \neq 0$, then let $x_1 = y_1 / x_0$. Otherwise, let $x_1$ be an arbitrary square root of $y_2$ and let $x_2 = y_3$. First, we see that $(x_0:x_1:x_2)$ is a valid point of $\P(1,1,2)$, because if all three of $x_0, x_1, x_2$ are zero, then $y_0 = x^2$ is zero, as is $y_2 = x_1^2$, and so is $y_1$, by $y_1^2 = y_0^2y_2^2$, and $y_3 = x_2 =0$. But at least one coordinate of $y$ must be nonzero, so $x$ must be nonzero as well.
		\par Finally, we see that $\Phi(x_0:x_1:x_2) = (y_0:y_1:y_2:y_3)$, which follows quickly from construction of $(x_0:x_1:x_2)$:
		\begin{align*}
			x_0^2 &= y_0\\
			x_0x_1 &= y_1\\
			x_1^2 &= y_2\\
			x_2 &= y_3
		\end{align*}
		\end{proof}
	\item Is $X$ a smooth variety? Prove or disprove your claim. Hint: look at $X$ in different affine charts in $\mathbb P^3$.
		 \begin{proof}
			 \textit{incomplete}
		 \end{proof}
\end{enumerate}
\end{problem}
\begin{problem}{3}
	Analagous to the case of projective space $\mathbb P^3$, write the affine charts in the weighted projective space $\mathbb P(1,1,2)$ and the gluing maps between them to show that $\mathbb P(1,1,2)$ can be realized as an abstract pre-variety.
	 \begin{proof}
		 Let $U_i$ be the set $x_i \neq 0 \subset \P(1,1,2)$. We will first consider the cases $i = 0, 1$, which are similar.
		 \par We see that the set $U_0$ is all those $(x_0:x_1:x_2)$ such that $x_0 \neq 0$. As noted above, we can renormalize by $x_0^{-\alpha_i/\alpha_0}$ to obtain coordinates $(1:x_1:x_2)$. Let $\varphi_0: U_0 \to \mathbb A^2$  be defined by
		 \begin{align*}
			 \varphi_0(x_1, x_2) = (1:x_1:x_2).\\
		 \end{align*}
		 which has an inverse morphism taking $(1:x_1:x_2)$ to $(x_1, x_2)$.
	 Similarly, we may define a map $\varphi_1: A^2 \to U_1$ by
	 \[\varphi_{1}( (x_0,x_2)) = (x_0: 1: x_2),\]
	 which has an inverse morphism taking $(x_0:1:x_2) \mapsto (x_0,x_2)$.
		 Both of these cases are effectively the same as in the projective case. However, the same map cannot be used for $U_2$, since $(x_0:x_1:1)$ is the same point as $(-x_0:-x_1:1)$, and there is no canonical choice of sign for the coordinates.
		 Instead, let $X \subset \mathbb A^3$ be the variety defined by the ideal $(y_0y_2 - y_1^2)$ of $k[y_0, y_1, y_2]$. We have a map $\varphi_2: X \to \mathbb A^2$ be defined by 
		 \[ \varphi_2(y_0:y_1:y_2) = (y_1:y_2:1).\]
		 This has an inverse morphism defined by taking 
	 \[(x_0:x_1:1) \to (x_0^2:x_0x_1:x_1^2)\]
	 The sets $U_i$ cover $\P(1,1,2)$, and the charts are all isomorphisms, so it remains to define the gluing maps, for intersections $U_i \cap U_j$.
	 \par First, for $U_0 \cap U_1$, we want to map the open subset $(\varphi_0^{-1}(\varphi_1(\mathbb A^2)) \subset \mathbb A^2$ to the open subset $(\varphi_1^{-1}(\varphi_0(\mathbb A^2))) \subset \mathbb A^2$. A point is in the first set if it is the image of a point $(a,b)$ which is first mapped to $(a:1:b)$, which then must be an element of $U_0$, meaning that $a \neq 0$. Then, $\varphi_0^{-1}$ takes this point to $(1/a, b/a^2)$. Since $a$ is arbitrary besides being nonzero, and $b$ is totally arbitrary, we see that the points of $(\varphi_0^{-1}(\varphi_1(\mathbb A^2))$ are exactly those $(x,y)$ where $x \neq 0$. Similarly, tracing elements, we see that the points of $(\varphi_1^{-1}(\varphi_0(\mathbb A^2))$ are also those $(x,y)$ where $x \neq 0$. So, the map $f: (x, y) \mapsto (1/x, y/x^2)$ is a well-defined regular map in both directions: we can set our gluing map $g_{ij} = g_{ji} = f $.
				 \par We do need to check that these maps give $\varphi_0 \circ g_{1,0} = \varphi_1 $, and vice versa. This can be seen by calculation:
				 \begin{align*}
					 \varphi_0(g_{1,0}(x,y)) &= \varphi_0(x,y)\\
					 &= (1:1/x:y/x^2)\\
					 &= (x: 1 : y)
					 &= \varphi_1(x,y)
				 \end{align*}
				 And, symmetrically, $\varphi_1 \circ g_{0,1} = \varphi_0$. Also, we see that $f \circ f = \text{Id}$.
				 \par Now, we want to define the gluing maps $g_{2i}$ and $g_{i2}$. We will define only the one for $i=0$, since the other case is very similar.
				 \par The gluing maps should be isomorphisms between the open set $\varphi_{2}^{-1}(\varphi_0(\A^2))$ and $\varphi_0^{-1}(\varphi_2(X))$. An element of the first set is of the form $(1/y, x/y, x^2/y)$, where $y$ is some nonzero value, and an element of the second set is of the form $(y^2 / (xy), 1/ (xy)^2)$, for some nonzero $xy$. 
				 \par So, we can define the gluing map $g_{2,0}$ on elements as
				 \[(x^2, xy, y^2) \mapsto ( xy/x^2,1/x^2 ) \]
				 and the gluing map $g_{0,2}$ on elements as
				 \[(a,b) \mapsto ( 1/(ab), 1/b,a/b).\]
				 These maps are indeed inverses, and satisfy $\varphi_0 \circ g_{2,0} = \varphi_2$, and $\varphi_2 \circ g_{0,2} = \varphi_0$. 
	 \end{proof}
\end{problem}
\begin{problem}{4}
	Find the Hilbert polynomial of $X = \mathbb P^n \times \mathbb P^m$ embedded in $\P^{(n+1)(m+1) - 1}$ via the Segre embedding. Find the degree of $X$ in $\P^{(n+1)(m+1)-1}$.
	\begin{proof}
		We recall that the Segre embedding of $\P^n \times \P^n$ in $\P^{(n+1)(m+1) - 1}$ is defined by sending
		\[ (a_0, \dots , a_n ) \times (b_0, \dots , b_m) \mapsto (a_0b_0, \dots , a_ibj , \dots a_nb_m\]
		Where the $a_ib_j$ are arranged in lexographic order on $(i,j)$. The degree-$t$ monomials $a_ib_j$ are exactly determined by a choice of $t$ $a_i$s and $t$ $b_j$s, where order does not matter, and repetitions are allowed. The way to choose $t$ elements fom a set of $k$, with repetitions, is exactly equal to $\binom {k + t - 1}{t} $. We have two choices of $t$ elements, from a set of $n + 1$ $a_i$s and $m + 1$ $b_j$s, so our total number of choices is $\binom{n + t}{t}\binom{m+t}{t}$. Any two of these monomials are unequal in the coordinate ring of $\P^n \times \P^m \subset \P^{(n+1)(m+1) - 1}$, so the dimension of the degree $t$ component of the coordinate ring is exactly equal to  $\binom{n + t}{t}\binom{m+t}{t}$. So, the Hilbert polynomial is
		\[H(t) = \binom{n + t}{t} \binom{m + t}{t}.\]
		To clean this up, we first use the identity $\binom a b = \binom a {a -b}$ to rewrite as
		\[H(t) = \binom{n + t}{n} \binom{m + t}{m}.\]
		We also want to use the identity $\frac{1}{n!m!} = \binom {n+m}{n} \frac{1}{(n+m)!}$.\\
		Now, writing $(x)_k = x(x-1)(x-2) \cdots (x-k+1)$, we have
		\begin{align*}
			H(t) &= \binom{n + t}{n} \binom{m + t}{m}\\
			&= \frac{(n+t)_n}{n!}\frac{(m+t)_m}{m!}\\
			&= \binom{n+m}{n} \frac{1}{(n+m)!} (n+t)_n (m+t)_m
			&= \binom{n+m}{n}\frac{1}{(n+m)!}( (t + n)(t + n - 1) \cdots (t+1)t)( (t+m)(t+m - 1) \cdots (t + 1) t)\\
			&= \binom{n+m}{n}\frac{1}{(n+m)!}(t^n + \cdots )(t^m + \cdots)\\
			&= \binom{n+m}{n}\frac{1}{(n+m)!}(t^{n+m}+\cdots)
		\end{align*}
		Since the leading coefficient of $H(t)$ is $\text{deg}(\P^n \times \P^m) / \text{dim}(\P^n \times \P^m)!$, we see that $\text{deg}(\P^n \times \P^m)$, for this embedding of $\P^n \times \P^m$ in $\P^{(n+1)(m+1)-1}$, is equal to $\binom{n+m}{n}$.
	\end{proof}
\end{problem}
\begin{problem}{5}
	Consider the affine plane curve $X = V(y^2 - x^3) \subset \mathbb A^2$ with coordinate ring $\mathbf k[X] = \mathbf k[x,y] / (y^2 - x^3)$. 
	\begin{enumerate}[label=(\alph*)]
		\item Show that $f = y/x \in k[X]$ is integral over $k[X]$.
			\begin{proof}
				We want to show that $y/x$ satisfies an integral polynomial in $k[X][f]$. In fact, it satisfies the equation $f^2 -x = 0$, since
				\begin{align*}
					f^2 - x &= \frac{y^2}{x^2} - x\\
					&= \frac{x^3}{x^2} - x\\
					&= x - x\\
					&= 0
				\end{align*}
				Since $y/x$ is not an element of $ k[x,y]/(x^2 - y^3) = k[X]$, the ring is not normal, and neither is the variety $X$. 
			\end{proof}
		\item Use $f$ to construct the normalization $\tilde X \subset \mathbb A^3$ and the normalization map $\pi: \tilde X \to X$ (verify that the $\tilde X$ you constructed is actually a normal variety.)
				\begin{proof}
					We want to extend our variety into a third dimension, by adjoining $t$: $\tilde X$ is the subset of $\mathbb A^3$ consisting of points $(x,y,t)$ which satisfy $x^3 = y^2$ and $t^2 = x$. 
					First, we see that the variety is normal: its coordinate ring $k[\tilde X]$ is $k[x,y,t]/(x^3 - y^2, t^2 - x)$. This is isomorphic to $k[t]$, since we can write any element of $k[\tilde X]$ in terms of $t$ alone by the identities $y = t^3$ and $x = t^2$. We know that $k[t]$ is normal, since it is a principal ideal domain, so $\tilde X$ is normal.
					\par Now, we show that the map $\pi: \tilde X \to X$ defined by taking $(x,y,t)$ to $(x,y)$ is a normalization - that is, it is a morphism which is finite and birational. It is birational by the map $X \backslash \left\{ 0 \right\} \to \tilde X$ defined by taking $(x,y)$ to $(x,y, x/y)$. The set $X \backslash \left\{ 0 \right\}$ is open and dense in $X$, so $\pi$ is a birational equivalence.
					\par Next, we want to show that $\pi$ is finite, i.e. that the induced map $k[X] \to k[\tilde X]$ makes $k[\tilde X]$ into a finitely generated $k[X]$ module. Since $k[\tilde X]$ is generated by $\left\{ t^i \right\}_{i \geq 0}$, it suffices to show that a finite number of these monomials generate the whole set. And in fact they do: for any $i \neq 1$, we can write $i$ as $2a + 3b$ for some nonnegative $a, b$, and so
					\begin{align*}
						t^i &= (t^{2a})(t^{3b})\\
						&= x^ay^b
					\end{align*}
					By replacing each $t^i$, for $i \neq 1$, with a monomial in $x$ and $y$, any $f \in k[\tilde X]$ can be written as $g(x,y) + kt$ for some $g(x,y) \in k[X]$. Thus the coordinate ring of $\tilde X$ is finitely genrated over the coordinate ring of $X$, and the map $\pi$ must be finite. As a finite birational morphism from a normal variety to $X$, it is the normalization of $X$.
				\end{proof}
	\end{enumerate}
\end{problem}
\begin{problem}{6}
	Assume we know the following fact: intersection of a projective variety $X \subset \P^N$ with a hyperplane $H$ in general position has dimension $\text{dim}(X) - 1$ (this is a corollary of the well-known Krull's principal ideal theorem of Hauptidealsatz). Prove the following:
	\begin{enumerate}[label=(\alph*)]
		\item Suppose $L \subset \P^N$ is a plane in general position with $\text{codim}(L) > \text{dim}(X)$. Show that $X \cap L$ is empty.
			\begin{proof}
				We recall that an arbitrary plane $L \subset \P^N$ can be written as the intersection of $\text{codim}(L)$-many hyperplanes $H$, or equivalently as $L' \cap H$, for $L'$ a plane with $\text{codim}(L') = \text{codim}(L) - 1$, and $H$ a hyperplane.
				\par We proceed by induction on $X$: if $X$ has dimension $0$, then $\text{codim}(L) > 1$, and $L = L' \cap H$ for some hyperplane $H$. But if $L \cap X$ is nonempty, we can make the following calculation:
				\begin{align*}
					\text{dim}(L \cap X) &= \text{dim}(L' \cap (H \cap X))\\
					&\leq \text{dim}(H \cap X)\\
					&= \text{dim}(X) - 1
					&= -1
				\end{align*}
				So in fact the set $L \cap X$ must be empty.
				\par Now, we make the inductive hypothesis that $L \cap X = \emptyset$ for all $\text{dim}(X) = n - 1$ and $\text{codim}(L) > \text{dim}(X)$. Let $X$ be a space of dimension $n$, and let $L$ be a plane in general postition with $\text{codim}(L) > \text{dim}(X)$. Then $L = L' \cap H$ for some plane $L'$ with $\text{codim}(L') = \text{codim}(L) - 1$ and some hyperplane $H$ in general position. Rewriting $L \cap X$ as $L' \cap (H \cap X)$, we see that $\text{dim}(H \cap X)$ has dimension $\text{dim}(X) - 1$, and $\text{codim}(L) = \text{codim}(L) - 1 > \text{dim}(X) - 1$. So, by the inductive hypothesis, the set $L \cap X$ is empty. By induction, the theorem holds for any projective variety $X \subset \P^N$.
			\end{proof}
		\item If $U \subset X$ is a nonempty open subset then $\text{deg}(U) = \text{deg}(X)$. Recall that degree is the number of intersection points with a plane of complementary dimension.
			 \begin{proof}
				 \textit{incomplete}
			 \end{proof}
	\end{enumerate}
\end{problem}
\maketitle
\end{document}
