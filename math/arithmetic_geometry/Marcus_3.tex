% Andrew Tindall
% --------------------------------------------------------------
% --------------------------------------------------------------
 
\documentclass[12pt]{article}
 
\usepackage[margin=1in]{geometry} 
\usepackage{amsmath,amsthm,amssymb,enumitem,hyperref,tikz-cd}

\newcommand{\N}{\mathbb{N}}
\newcommand{\Q}{\mathbb{Q}}
\newcommand{\Z}{\mathbb{Z}}
\newcommand{\C}{\mathbb{C}}
\newcommand{\R}{\mathbb{R}}
\newcommand{\mc}[1]{\mathcal{#1}}
\newcommand{\e}{\varepsilon}
\newcommand{\bs}{\backslash}
\newcommand{\PGL}{\text{PGL}}
\newcommand{\Sp}{\text{Sp}}
\newcommand{\tr}{\text{tr}}
\newcommand{\Lie}{\text{Lie}}
\newcommand{\rec}[1]{\frac{1}{#1}}
\newcommand{\toinf}{\rightarrow \infty}


\theoremstyle{definition}
\newtheorem{proofpart}{Part}
\newtheorem{theorem}{Theorem}
\makeatletter
\@addtoreset{proofpart}{theorem}
\makeatother


\newenvironment{problem}[2][Problem]{\begin{trivlist}
\item[\hskip \labelsep {\bfseries #1}\hskip \labelsep {\bfseries #2.}]}{\end{trivlist}}
 
\begin{document}
 
%\renewcommand{\qedsymbol}{\filledbox}
 
\title{Marcus Chapter 3 Exercises}
\author{Andrew Tindall}
 
\maketitle
\begin{problem}{4}
	Let $K$ be a number field of degree $n$ over $\Q$. Prove that every nonzero ideal $I$ in $R = \mathbb A \cap K$ is a free abelian group of rank $n$. (Hint: $\alpha R \subset I \subset R$ for any $\alpha \in I$, $\alpha \neq 0$.)
	\begin{proof}
		We know from Marcus that $R = \mathbb A \cap K$ is a free abelian group of rank $n$; we also see that $\alpha R \simeq \R$ as an abelian group, by the correspondence $\alpha r \mapsto r $, which is a homomorphism of abelian groups because $\alpha(r_1 + r_2) = \alpha(r_1) + \alpha(r_2)$. 
		Finally, we know that any subgroup of a free abelian group is also a free abelian group, of lesser or equal rank. So, we can sandwich the ideal $I$ between two abelian groups:
		\[ \Z^n \leq I \leq \Z^n\]
		So, we see that $I \simeq \Z^r$, where $r \leq n$ and $n \leq r$. Indeed, $I$ is a free abelian group of rank $n$.
	\end{proof}
\end{problem}
\begin{problem}{8}
	\begin{enumerate}[label=(\alph*)]
		\item Show that the ideal $(2, x)$ in $\Z[x]$ is not principal.
			\begin{proof}
				The elements of $(2, x)$ are exactly the polynomials with integer coefficients such that the constant term is even.
				If $(2,x)$ were principal, say $(2, x) = (f)$, then there are two possibilities: if $f$ is of degree $0$, then it has an even constant term, and every element of $(f)$ will have all even coefficients.
				But $x \in (2, x)$, and it does not have even coefficients.
				If $f$ is of degree $\geq 1$, then every element of $(f)$ will also have degree at least $1$.
				But $2 \in (2, x)$, and it has degree $0$. So, $(2, x)$ must not be princpal.
			\end{proof}
		\item Let $f, g \in \Z[x]$ and let $m$ and $n$ be the gcd's of the coefficients of $f$ and $g$, repsectively.
			Prove Gauss' Lemma: $mn$ is the gcd of the coefficients of $fg$. (Hint: Reduce to the case in which $m = n = 1$ and argue as in the lemma for Theorem $1$.)
			\begin{proof}
				First, we note that if $mn$ is not $1$, then we can factor it out of $fg$.
				We have $f = mf_1$, for some $f_1 \in \Z[x]$, where the gcd of the coefficients of $f_1$ is $1$.
				Similarly $g = ng_1$ for some $g_1 \in \Z[x]$, with the gcd of the coefficients of $g_1$ equal to $1$.
				Then we have $fg = nm(f_1g_1)$, and the gcd of the coefficients of $fg$ is $nm$ times the gcd of the coefficients of $f_1g_1$.
				So, it will suffice to assume that $n = m = 1$ and show that the of the coefficients of $fg$ is $1$.
				\par Assume for the sake of contradiction that the gcd of the coefficients of $fg$ is not $1$. 
				Let $p$ be a prime dividing the coefficients of $fg$; then $fg = ph$ for some $h \in \Z[x]$.
				Reducing coefficients modulo $p$, we have $\overline f \overline g = 0$ in $\Z_p[x]$.
				But $\Z_p[x]$ is an integral domain, so either $\overline f$ or $\overline g$ is $0$, and all of its coefficients are divisible by $p$, contradicting the assumption that their gcd was $1$.
			\end{proof}
		\item Use (b) to show that if $f \in \Z[x]$ and $f$ is irreducible over $\Z$, then $f$ is irreducible over $\Q$.
			(We already know this for monic polynomials.)
			\begin{proof}
				<++>
			\end{proof}
		\item Suppose $f$ is irreducible over $\Z$ and the gcd of its coefficients is $1$. Show that if $f \mid gh$ in $\Z[x]$, then $f \mid g$ or $f \mid h$ (Use (b) and (c)).
			\begin{proof}
				<++>
			\end{proof}
		\item Show that $\Z[x]$ is a UFD, the irreducible elements being the polynomials $f$ as in (d), along with the primes $p \in \Z$.
			\begin{proof}
				<++>
			\end{proof}
	\end{enumerate}
				We know that if $I$ is prime, then $I = (I \cap R) S$ if and only if it has ramification index $1$ over $(I \cap R)$, and is the unique prime lying over $I \cap R$. By the formula $ref = n$, where $r$ is the number of primes lying over $I \cap R$, $e$ is the ramification index of each prime, and $f$ is the inertial degree of each prime, we see that $r = e = 1$ whenever $f = n$; i.e. whenever the inertial degree of $I$ over $I \cap R$ is equal to the extension degree of $L$ over $K$.
\end{problem}
\begin{problem}{9}
Let $K$ and $L$ be number fields, $K \subset L$, $R = \mathcal A \cap K$, $A = \mathcal A \cap L$.
\begin{enumerate}[label=(\alph*)]
		\item Let $I$ and $J$ be ideals in $R$, and suppose $IS \mid JS$. show that $I \mid J$. (Suggestion: Factor $I$ and $J$ into primes in $R$ and consider what happens in $S$.)
			\begin{proof}
				We can factor $I$ and $J$ uniquely into prime ideals in $R$- say, $I = \Pi_{1 \leq i \leq n} P_i^{e_i}$ and $J = \Pi_{1 \leq j \leq m}Q_{j}^{e_j}$.
Then  
			\end{proof}
		\item Show that for each ideal $I$ in $R$, we have $I = IS \cap R$. (Set $J = IS \cap R$ and use (a).)
			\begin{proof}
				<++>
			\end{proof}
		\item Characterize those ideals $I$ of $S$ such that $I = (I \cap R) S$. 
			\begin{proof}
			\end{proof}
	\end{enumerate}
\end{problem}
\begin{problem}{<++>}
<++>
\end{problem}<++>
\end{document}
